\documentclass[11pt]{article}

\begin{document}

{
\begin{center}
{\Large VLBA Pipeline: Outline of Data Reduction Heuristics} \\
~ \\
Gareth Hunt, Bill Cotton, \& Jared Crossley \\
29 February 2012
\end{center}}

\section{Introduction}

The VLBA Pipeline was designed to take uncalibrated VLBA visibility data
directly from the NRAO archive and to create a file set for
reingestion into the archive.  
This file set contains reference images with associated diagnostic
plots, reports, scripts, and log files, plus calibrated visibility data
with associated tables.
The scripts can be used to set non-default values to processing
parameters and used to repeat part or all of the processing if the
default processing is inadequate.


\subsection{Scope}

The scope of the present version of the pipeline is:

\begin{itemize}
\item VLBA data only\\
It may work with the inclusion of other telescopes if all of the VLBA
calibration tables are available.
\item 1-15 GHz\\
The pipeline has been used on continuum data sets with frequencies as
high as 43 GHz with robust results.
\item Calibrated fluxes\\
Calibration uses standard external calibration and does not include
coherence losses.
\item Continuum imaging\\
Spectral line data sets can have the continuum calibration done but
no spectral cubes are made.
\item Imaging including self-calibration\\
Multi--resolution imaging with self calibration is done.
\item No polarization \\
No polarization calibration/imaging is currently implemented.
\end{itemize}

\subsection{Software}

The VLBA pipeline is:

\begin{itemize}
\item Written in python, and
\item Uses Obit and AIPS tasks to do the data processing, and
\item Uses AIPS data structures for intermediate data, and
\item Writes FITS images and (AIPS FITAB format) calibrated datasets.
\end{itemize}

The pipeline scripts are available for checkout from a Subversion
(SVN) repository (https://svn.cv.nrao.edu/svn/VLBApipeline).

AIPS (http://www.aips.nrao.edu/index.shtml) and

Obit (http://www.cv.nrao.edu/~bcotton/Obit.html)

are installed on all NRAO Linux computers and available for 
installation via download to non-NRAO computers.

\subsection{Prototype Comparison}

The Mojave project was selected as an initial set of observations.  This
comprises more than 150 datasets, each roughly 24 hours in duration, observing
sources to track morphological changes over time.  The observations are
snapshots mostly at 2cm (Ku-band) with some 3cm (X-band) observations.

This extended project has the advantage that the data have already
been calibrated and imaged by experts, and that the resultant images
are publicly available for direct comparison with the images produced
by the pipeline. 

The FITS images of the Mojave project are available at

http://www.nrao.edu/2cmsurvey/.

For consistency between epochs, the Mojave project necessarily has limitations
on the data that they fully reduce.  The VLBA pipeline has no such limitations,
and about 3500 individual images were produced from data taken between August
2003 and December 2011.  These images typically had dymanic ranges (peak:rms)
of 25-35dB.  Roughly 3200 images were available for direct comparison.  The
comparison was excellent.  On average, the integrated fluxes for the pipeline
were just over 5\% lower than the images in the Mojave catalog, as predicted.

\section{The Process}

The pipeline processing uses the following processes.
Many of the default processing parameters are frequency dependent and
may be overridden and the various steps may be turned on or off.

\begin{enumerate}
\item Data retrieved from the archive\\
Pre-DifX data may be either multiple FITS IDI format files or a single
AIPS UVFITS data file.
Data from the DifX correlator are in a single FITS IDI format file.
\item Data converted to AIPS format\\
Multiple  FITS IDI format files can be concatenated.
\item ``Quack''\\
The first few seconds of each scan is flagged.
\item Initial data filtering \\
The data are edited with a running median window to flag deviant data
such as when an antenna is late on source.
\item Standard ``external'' calibration
\begin {enumerate}
\item 1/2 bit sampling correction\\
Uses AIPS task ACCOR.
\item Parallactic angle correction\\
Phases are corrected for the effects of parallactic angle.
Uses AIPS task CLCOR.
\item Tsys/atmosphere/gain correction\\
The amplitudes are converted to Jy using measured system temperatures,
standard gain curves and atmospheric opacity corrections estimated
from the system temperatures.
Uses AIPS task APCAL.
These gains are smoothed before application to the data.
\item Calibrator selection\\
``Calibrator'' sources are then determined by doing a fringe fit on
all sources to determine which ones reliably give solid detections.
The reference antenna is picked on the basis of strong source
detections.
The best calibration scan is then selected on the basis of the fringe
fit signal--to--noise estimates.
This scan is the one involving the largest number of antennas and with
the highest average SNR.
Obit task Calib is  used for the fringe fitting.
\item Pulse calibration\\
The pulse cal signals are used to align the phases and delays of the
various parts of the electronics.
Since these are based on phase measurements from discrete tones, the
delays are ambiguous.
This ambiguity is resolved using fringe fit results for the ``best''
calibrator scan.
Obit tasks PCCor + CLCal are used for this.
\item ``Manual'' phase calibration\\
There are generally residuals delay and phase errors after correction
by the pulse calibration;  these are corrected using delays and phases
determined for the ``best'' calibrator scan and applied to all data.
Obit tasks Calib + CLCal are used for this
\end{enumerate}
\item Calibration from visibility data.
\begin {enumerate}
\item Initial calibrator self--calibration\\
All sources deemed to be calibrators are self calibrated to provide
initial images for further calibration.
Phase calibration is applied and amplitude as well if the peak in the
image exceeds a frequency dependent minimum value.
Imaging uses Obit task SCMap.
\item Delay calibration\\
Group delay fits are made using a fringe fit on the calibrator sources
using the source models derived in the previous step.
Obit tasks Calib  + CLCal are used for the fringe fitting.
\item Bandpass correction \\
A bandpass correction for the amplitudes and phases in each channel is
determined from the best calibrator scan and the model derived for
that calibrator from the cross--correlation data.
No spectral index correction is included.
Uses Obit task BPass.
\item Calibrator phase calibration \\
Phase corrections on a short time scale are determined for the
calibrator sources using the source models for each.
This phase correction is then applied to the data (needed in the next
step).
Obit tasks Calib  + CLCal are used.
\item Calibrator amplitude calibration \\
Longer time amplitude solutions are determined for the calibrator
sources. 
The strong enough calibrator sources have the solutions determined for
them applied in the calibration table.
Other sources use a smoothed version of the amplitude calibration
solutions. 
\item Calibrate and average data. \\
Calibration is applied and the data are averaged in frequency and
possibly time.
Subsequent steps use the averaged data.
Uses Obit task Splat.
\item Self calibration of all sources \\
An initial self calibration to get models of all sources is
performed. 
Phase self--cal is always used and also amplitude self--cal if the
peak in the image is above a given threshold.
Imaging uses Obit task SCMap.
\item Data clipping \\
Data with amplitudes significantly in excess of the sum of the CLEAN
components for each source are flagged.
\item Phase calibration of all sources\\
The source models are used to determine the phase corrections for all
sources and these are applied to the cumulative calibration table.
Obit tasks Calib  + CLCal are used.
\end {enumerate}
\item Imaging and production of results.
\begin {enumerate}
\item Imaging\\
Each source for which previous calibration was successful is then
imaged.
This final imaging may use phase and possibly amplitude
self--calibration and the imaging uses multiple resolutions(2) to help
recover extended emission.
Obit task Imager is used for the imaging.
\item Saving images\\
Final and calibration images are written to FITS files.
\item Saving visibility data\\
The averaged and calibrated uv data and the tables from the initial
data are written to AIPS FITAB format FITS files.
\item Reports\\
Statistics of the images are determined and a HTML page constructed to
simplify viewing the results.
An XML file manifest is generated for re-ingestion into the archive.
\item Cleanup\\
All AIPS data files are deleted.
\end {enumerate}
\end {enumerate}

\section {The Products}

\begin{itemize}
\item Calibrated (u,v) dataset with calibration and flagging tables in
AIPS FITAB format -- Tables from initial data and averaged
visibilities per input dataset. 
\item FITS Images -- one per source observed plus calibration images.
\item Diagnostic plots -- several per image.
\item Reports and logs created during the process
\item Meta-data for a VOTable to describe the products
\end{itemize}

The file set comprising all files and the meta-data are stored in a single
directory.  For approved pipeline use, this directory is stored on the lustre
file system in NRAO Socorro.  From there it is ingested directly into the NRAO
archive.

Sources that did not image acceptably are added to failTargets list.  This is
referenced in the HTML Report.

\clearpage
\appendix
\section{General Parameters}

This section lists the default global parameters used in the VLBA Pipeline
scripts.  They are only explained briefly, but experienced users should have no
difficulty recognizing their use and functionality.  It is clearly possible to
re-run or re-start the pipeline using different values than the defaults.

Several parameters are actually placeholders for derived intermediate products:
failTarg, contCalModel, targetModel; although, in principle, contCalModel could
be user-supplied.  These are initialized as specified here at the beginning of
the pipeline process.

\vfill
\begin{center}
\begin{tabular}{|l|c|l|}

\hline
\multicolumn{3}{|l|}{Quantization correction} \\
\hline
doQuantCor & True & Do quantization correction \\
QuantSmo & 0.5 & Smoothing time (hr) for quantization corrections \\
QuantFlag & 0.0 & If $>$0, flag solutions $<$ QuantFlag \\
 & & (use 0.9 for 1 bit, 0.8 for 2 bit) \\

\hline
\multicolumn{3}{|l|}{Parallactic angle correction} \\
\hline
doPACor & True & Make parallactic angle correction \\

\hline
\multicolumn{3}{|l|}{Opacity/Tsys correction} \\
\hline
doOpacCor & True & Make Opacity/Tsys/gain correction? \\
OpacSmoo & 0.25 & Smoothing time (hr) for opacity corrections \\

\hline
\multicolumn{3}{|l|}{Special editing list} \\
\hline
doEditList & False & Edit using editList? \\
editFG & 2 & Table to apply edit list to \\
editList & [] & EditList \\

\hline
\multicolumn{3}{|l|}{Do median flagging} \\
\hline
doMedn & True & Median editing? \\
mednSigma & 10.0 & Median sigma clipping level \\
mednTimeWind & 1.0 & Median window width in min for median flagging \\
mednAvgTime & 10.0/60. & Median Averaging time in min \\
mednAvgFreq & 0 & Median 1=$>$avg chAvg chans, 2=$>$avg all chan, \\
 & & 3=$>$ avg chan and IFs \\
mednChAvg & 1 & Median number of channels to average \\

\hline
\multicolumn{3}{|l|}{Apply phase cal corrections?} \\
\hline
doPCcor & True & Apply PC table? \\
doPCPlot & True & Plot results? \\

\hline
\end{tabular}
\end{center}
\clearpage
\begin{center}
\begin{tabular}{|l|c|l|}

\hline
\multicolumn{3}{|l|}{``Manual" phase cal - even to tweak up PCals} \\
\hline
doManPCal & True & Determine and apply manual phase cals? \\
manPCsolInt & None & Manual phase cal solution interval (min) \\
manPCSmoo & None & Manual phase cal smoothing time (hr) \\
doManPCalPlot & True & Plot the phase and delays from manual phase cal \\

\hline
\multicolumn{3}{|l|}{Bandpass Calibration?} \\
\hline
doBPCal & True & Determine Bandpass calibration \\
bpBChan1 & 1 & Low freq. channel,initial cal \\
bpEChan1 & 0 & Highest freq channel, initial cal, 0=$>$all \\
bpDoCenter1 & None & Fraction ofchannels in 1st, overrides bpBChan1, \\
 & & bpEChan1 \\
bpBChan2 & 1 & Low freq. channel for BP cal \\
bpEChan2 & 0 & Highest freq channel for BP cal,0=$>$all  \\
bpChWid2 & 1 & Number of channels in running mean BP soln \\
bpdoAuto & False & Use autocorrelations rather than cross? \\
bpsolMode & `A\&P' & Band pass type `A\&P', `P', `P!A' \\
bpsolint1 & None & BPass phase correction solution in min \\
bpsolint2 & 10.0 & BPass bandpass solution in min \\
specIndex & 0.0 & Spectral index of BP Cal \\
doSpecPlot & True & Plot the amp. and phase across the spectrum \\

\hline
\multicolumn{3}{|l|}{Editing} \\
\hline
doClearTab & True & Clear cal/edit tables \\
doGain & True & Clear SN and CL tables $>$1 \\
doFlag & True & Clear FG tables $>$ 1 \\
doBP & True & Clear BP tables? \\
doCopyFG & True & Copy FG 1 to FG 2quack \\
doQuack & True & Quack data? \\
quackBegDrop & 0.1 & Time to drop from start of each scan in min \\
quackEndDrop & 0.0 & Time to drop from end of each scan in min \\
quackReason & ``Quack" & Reason string \\

\hline
\multicolumn{3}{|l|}{Amp/phase calibration parameters} \\
\hline
refAnt & 0 & Reference antenna \\
refAnts & [0] & List of Reference antenna for fringe fitting \\

\hline
\end{tabular}
\end{center}
\clearpage
\begin{center}
\begin{tabular}{|l|c|l|}

\hline
\multicolumn{3}{|l|}{Imaging calibrators (contCals) and targets} \\
\hline
doImgCal & True & Image calibrators \\
targets & [] & List of target sources \\
failTarg & [] & List of failed target (source,process) \\
doImgTarget & True & Image targets? \\
outCclass & ``ICalSC" & Output calibrator image class \\
outTclass & ``IImgSC" & Output target temporary image class \\
outIclass & ``IClean" & Output target final image class \\
Robust & 0.0 & Weighting robust parameter \\
FOV & None & Field of view radius in deg. \\
Niter & 500 & Max number of clean iterations \\
minFlux & 0.0 & Minimum CLEAN flux density \\
minSNR & 4.0 & Minimum Allowed SNR \\
solMode & ``DELA" & Delay solution for phase self cal \\
avgPol & True & Average poln in self cal? \\
avgIF & False & Average IF in self cal? \\
maxPSCLoop & 6 & Max. number of phase self cal loops \\
minFluxPSC & 0.05 & Min flux density peak for phase self cal \\
solPInt & None & phase self cal solution interval (min) \\
maxASCLoop & 1 & Max. number of Amp+phase self cal loops \\
minFluxASC & 0.2 & Min flux density peak for amp+phase self cal \\
solAInt & None & amp+phase self cal solution interval (min) \\
nTaper & 1 & Number of additional imaging multiresolution tapers \\
Tapers & [20.0,0.0] & List of tapers in pixels \\
do3D & False & Make ref. pixel tangent to celest. sphere for each facet \\
% (SET TO FALSE AS WORKAROUND TO CC TABLE BUG!)
noNeg & False & F=Allow negative components in self cal model \\

\hline
\multicolumn{3}{|l|}{Find good calibration data} \\
\hline
doFindCal & True & Search for good calibration/reference antenna \\
findSolInt & None & Solution interval (min) for Calib \\
findTimeInt & None & Maximum timerange, large=$>$scan \\
contCals & None & Name or list of continuum cals \\
contCalModel & None & No cal model \\
targetModel & None & No target model yet \\

\hline
\multicolumn{3}{|l|}{If need to search for calibrators} \\
\hline
doFindOK & True & Search for OK cals if contCals not given \\
minOKFract & 0.5 & Minimum fraction of acceptable solutions \\
minOKSNR & 20.0 & Minimum test SNR \\
failTarg & [] & list of failed sources \\

\hline
\end{tabular}
\end{center}
\clearpage
\begin{center}
\begin{tabular}{|l|c|l|}

\hline
\multicolumn{3}{|l|}{Delay calibration} \\
\hline
doDelayCal & True & Determine/apply delays from contCals \\
delaySmoo & None & Delay smoothing time (hr) \\

\hline
\multicolumn{3}{|l|}{Amplitude calibration} \\
\hline
doAmpCal & True & Determine/smooth/apply amplitudes \\
 & & from contCals \\

\hline
\multicolumn{3}{|l|}{Apply calibration and average?} \\
\hline
doCalAvg & True & calibrate and average cont. calibrator data \\
avgClass & ``UVAvg" & AIPS class of calibrated/averaged uv data \\
CalAvgTime & None & Time for averaging calibrated uv data (min) \\
CABIF & 1 & First IF to copy \\
CAEIF & 0 & Highest IF to copy \\
CABChan & 1 & First Channel to copy \\
CAEChan & 0 & Highest Channel to copy \\
chAvg & 10000000 & Average all channels \\
avgFreq & 1 & Average all channels \\

\hline
\multicolumn{3}{|l|}{Phase calibration of all targets in averaged calibrated
  data} \\
\hline
doPhaseCal & True & Phase calibrate contCals data with self-cal? \\
doPhaseCal2 & True & Phase target data with self-cal? \\

\hline
\multicolumn{3}{|l|}{Instrumental polarization cal?} \\
\hline
doInstPol & False & determination instrumental polarization \\
 & & from instPolCal \\
instPolCal & None & Defaults to contCals \\

\hline
\multicolumn{3}{|l|}{Right-Left phase (EVPA) calibration} \\
\hline
doRLCal & False & Set RL phases from RLCal - also needs RLCal \\
RLCal & None & RL Calibrator source name, if given, a list of triplets, \\
 & & (name, R-L phase (deg@1GHz), RM (rad/$m^{2}$)) \\

\hline
\multicolumn{3}{|l|}{Clip excessive visibilities} \\
\hline
doClipFlag & True & Clip (flag) visibilities above sum of CCs? \\
clipFactor & 1.25 & Factor above sum of CCs to clip \\
clipTime & 0.25 & Time in min for which the data is to be averaged \\
 & & before clipping \\

\hline
\multicolumn{3}{|l|}{Final Image/Clean} \\
\hline
doImgFullTarget & True & Final Image/Clean/selfcal \\
Stokes & ``I'' & Stokes to image \\
doKntrPlots & True & Contour plots \\

\hline
\end{tabular}
\end{center}
\clearpage
\begin{center}
\begin{tabular}{|l|c|l|}

\hline
\multicolumn{3}{|l|}{Final} \\
\hline
outDisk & 0 & FITS disk number for output (0=cwd) \\
doSaveUV & True & Save uv data \\
doSaveImg & True & Save images \\
doSaveTab & True & Save Tables \\
doCleanup & True & Destroy AIPS files \\
copyDestDir & `' & Destination directory for copying output files \\
 & & empty string -$>$ do not copy \\

\hline
\multicolumn{3}{|l|}{Diagnostics} \\
\hline
doSNPlot & True & Plot SN tables etc \\
doDiagPlots & True & Plot single source diagnostics \\
prtLv & 2 & Amount of task print diagnostics \\
doMetadata & True & Save source and project metadata \\
doHTML & True & Output HTML report \\
\hline

\end{tabular}
\end{center}

\vfill
\section{Band-dependent Parameters}

This section lists the default band-dependent parameters used in the VLBA
Pipeline scripts.  They are only explained briefly, but experienced users
should have no difficulty recognizing their use and functionality.  It is
clearly possible to re-run or re-start the pipeline using different values than
the defaults.

Note that the VLBA has two receiver bands below 1GHz (90cm and 50cm).  The
band-dependent parameters are the same for both bands.  Note also that 9mm (Ka)
is included for completeness in the software, but there is no receiver.

\clearpage
\begin{center}
\begin{tabular}{|l|l|}

\hline
Parameter & Description \\
\hline
manPCsolInt & Manual phase cal solution interval (min) \\
manPCSmoo & Manual phase cal smoothing time (hr) \\
delaySmoo & Delay smoothing time (hr) \\
bpsolint1 & BPass phase correction solution in min \\
FOV & Field of view radius in deg. \\
solPInt & phase self cal solution interval (min) \\
solAInt & amp+phase self cal solution interval (min) \\
findSolInt & Solution interval (min) for Calib \\
findTimeInt & Maximum timerange, large=$>$scan \\
CalAvgTime & Time for averaging calibrated uv data (min) \\
\hline

\end{tabular}
\end{center}

\begin{center}
\begin{tabular}{|l|c|c|c|c|c|}

\hline
Parameter     & $<$1GHz (P) & 20cm (L)  & 13cm (S)  & 6cm (C)  & 3cm  (X) \\
\hline
manPCsolInt & 0.25         & 0.5          & 0.5           & 0.5           & 0.5          \\
manPCSmoo & 10.0         & 10.0         & 10.0         & 10.0         & 10.0       \\
delaySmoo   &  0.5           & 0.5          & 0.5           & 0.5           & 0.5          \\
bpsolint1     & 10/60       & 15/60      & 10/60      & 10/60      & 10/60      \\
FOV              & 0.4/3600  & 0.4/3600 & 0.2/3600 & 0.2/3600 & 0.1/3600 \\
solPInt         & 0.10          & 0.25         & 0.25         & 0.25        & 0.25         \\
solAInt         & 3.0            & 3.0          & 3.0          & 3.0           & 3.0           \\
findSolInt     & 0.1           & 0.25         & 0.25         & 0.5         & 0.5         \\
findTimeInt  & 10.0          & 10.0        & 10.0        & 10.0        & 10.0        \\
CalAvgTime & 10/60       & 10/60      & 10/60      & 10/60      & 10/60     \\
\hline

\end{tabular}
\end{center}
\begin{center}
\begin{tabular}{|l|c|c|c|c|c|}

\hline
Parameter     & 2cm (Ku)  & 1cm (K)   & 9mm (Ka) & 7mm (Q)  & 3mm (W)   \\
\hline
manPCsolInt & 0.5           & 0.2         & 0.2           & 0.1           & 0.1           \\
manPCSmoo & 10.0         & 10.0        & 10.0         & 10.0        & 10.0        \\
delaySmoo   & 0.5           & 0.5           & 0.5          & 0.5           & 0.5          \\
bpsolint1     & 10/60      & 10/60      & 10/60      & 5/60      & 5/60      \\
FOV              & 0.05/3600 & 0.05/3600 & 0.04/3600 & 0.04/3600 & 0.02/3600 \\
solPInt         & 0.25         & 0.25         & 0.25         & 0.1        & 0.1        \\
solAInt         & 3.0          & 3.0           & 3.0           & 3.0           & 3.0          \\
findSolInt     & 0.5         & 0.3           & 0.2            & 0.1          & 0.1         \\
findTimeInt  & 10.0        & 10.0         & 10.0         & 10.0         & 10.0        \\
CalAvgTime & 10/60      & 5/60       & 5/60         & 5/60       & 4/60      \\
\hline

\end{tabular}
\end{center}

\end{document}
