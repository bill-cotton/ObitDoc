\documentclass[11pt]{article}

\begin{document}

{
\begin{center}
{\Large Obit Development Memo 40\\ }
{\Large ALMA Continuum Scripts: \\
Outline of Data Reduction and Heuristics} \\
~ \\
Bill Cotton \\
\today
\end{center}}

\section{Introduction}



\subsection{Scope}

The scope of the present version of the ALMA continuum scripts is to
perform standard calibration and editing of ALMA data and produce
continuum, possibly wide-band, images of target sources.
Logs, reports and numerous diagnostic plots help evaluate the results
of the processing.
If default processing parameters are adequate, the scripts will start
from ALMA archive ASDM/BDF files and result in FITS images,
calibrated data, reports, plots etc.
The scripting is also capable of being highly tuned to a particular
project and can be rerun in whole or part with user specified
parameters.

\subsection{Software}

The ALMA Obit scripts are:

\begin{itemize}
\item Written in python, and
\item Use Obit and AIPS tasks to do the data processing, and
\item Use AIPS data structures for intermediate data, and
\item Write FITS images and (AIPS FITAB format) calibrated data-sets.
\end{itemize}

AIPS (http://www.aips.nrao.edu/index.shtml) and\\
Obit (http://www.cv.nrao.edu/~bcotton/Obit.html)
are installed on all NRAO Linux computers and available for 
installation via download to non-NRAO computers.
A binary distribution is supported for Linux.
The ALMA scripts are in the \$OBIT/python directory with the template
parameter script in \$OBIT/share/scripts.

\section{Execution}
Several steps are needed to execute the ALMA scripts.
\subsection{Generate parameter scripts}
The processing is guided by values in python parameter scripts.
These scripts can be created and initialized by information gleaned from
the ALMA archive ALMA Science Data Model (ASDM) files using routine
ALMACal.ALMAPrepare (see section \ref{Details}).
This will create one or more parameter files, each of which needs to
be processed separately.

Alternatively, the parameter file can be derived manually using the
template file \$OBIT/share/scripts/ALMATemplateParm.py and making
the substitutions described in the file.

\subsection{Modify parameter scripts}
If the default values in the automatically generated parameter
script(s) are not appropriate, they can be changed, see Section
\ref{Tune}. 
The details of each processing step and the parameters used are
described in Section \ref{Details}.
Default parameters and control switches can be overridden in the
parameter scripts.
Additional calibrator model information can be entered as described in
section \ref{calmodel}.
The end of the parameter script contains switches to turn on and off
various stages.

\subsection{AIPS and Obit setup scripts}
A script needs to be created giving the details of the AIPS and Obit
installations. 
This script is described in detail in Section \ref{Setup}.

\subsection{Execute scripts}
Each of the parameter scripts can be executed from the Unix shell by
\begin{verbatim}
 > ObitTalk ALMAContPipe.py AIPSSetup.py \
      ALMAContParm_myProject_Cfg0\Nch64.py
\end{verbatim}
where ALMAContParm\_myProject\_Cfg0\_Nch64.py is the name of your
parameter script.

This procedure can start from an archive data set and result in a
set of calibrated data, images, reports, logs and various diagnostic
plots, see Section \ref{products} for details.

\section{Calibrator models}\label{calmodel}
All standard, constant calibrator models are weak and resolved at
mm/submm wavelengths but may still be usable at lower frequencies.
Some QSOs are monitored and point models may be adequate.
Solar system objects are frequently used for flux calibrators;
the calibration process will include self calibration which will
derive a model, the parameter CalModelFlux can be used to give the
total flux density.

The ALMA calibration scripts operate on arrays of calibrator dict
structures with the following entries.
These allow specifying parameterized models or images with CLEAN
components (AIPS or FITS)
\begin{itemize}
\item{\tt  Source}       Source name as given in the SU table. 
\item{\tt  CalFile}      Calibrator model Cleaned FITS file name
\item{\tt  CalName}      Calibrator model Cleaned AIPS  map name 
\item{\tt  CalClass}     Calibrator model Cleaned AIPS  map class
\item{\tt  CalSeq}       Calibrator model Cleaned AIPS  map seq
\item{\tt  CalDisk}      Calibrator model Cleaned AIPS  map disk
\item{\tt  CalNfield}    Calibrator model No. maps to use for model
\item{\tt  CalCCVer}     Calibrator model CC file version
\item{\tt  CalBComp}     Calibrator model First CLEAN comp to use, 1/field
\item{\tt  CalEComp}     Calibrator model Last CLEAN comp to use, 0=>all
\item{\tt  CalCmethod}   Calibrator model Modeling method, 'DFT','GRID','    '
\item{\tt  CalCmodel}    Calibrator model Model type: 'COMP','IMAG'
\item{\tt  CalFlux}      Calibrator model Lowest CC component used
\item{\tt  CalModelSI}   Calibrator Spectral index
\item{\tt  CalModelFlux} Parameterized model flux density (Jy)
\item{\tt  CalModelPos}  Parameterized model Model position offset (asec)
\item{\tt  CalModelParm} Parameterized model Model parameters (maj, min, pa, type)
\item{\tt  useSetJy}     If True, use Flux density given by task SetJy
\end{itemize}
These dicts are created in the parameter script by routine
ALMACal.ALMACalModel for the various types of calibrators.
ALMACal.ALMAStdModel is then used to fill in the details about
calibrators it knows about and can find in the first FITS directory;
unfortunately, this is not much for ALMA.
Information not known to these scripts may be entered into the
calibrator dict structure in the parameter script.

\section{AIPS and Obit Setup}\label{Setup}
These scripts use data in AIPS format and some AIPS tasks; the
location of the AIPS data directories and other details as well as the
Obit initialization are given in the AIPSSetup.py file.
The items that need to be specified are:
\begin{itemize}
\item {\tt adirs} \\
A list of the AIPS data directories as a tuple,  the first element is
the URL of the ObitTalkServer or None for local disk.
The second element is the directory path.
\item {\tt fdirs} \\
A list of the FITS data directories as a tuple,  the first element is
the URL of the ObitTalkServer or None for local disk.
The second element is the directory path.
\item {\tt user} \\
The AIPS user number to be used.
\item {\tt AIPS\_ROOT} \\
The root of the AIPS system directories.
An environment variable of this name is set by the AIPS startup scripts.
Python None will default to your AIPS setup.
\item {\tt AIPS\_VERSION} \\
The  AIPS version.
An environment variable of this name is set by the AIPS startup scripts.
Python None will default to your AIPS setup.
\item {\tt DA00} \\
The  AIPS DA00 directory (TDD000004; file needed).
An environment variable of this name is set by the AIPS startup
scripts.
Python None will default to your AIPS setup.
\item {\tt OBIT\_EXEC} \\
The root directory of your Obit directories.
Python None will default to your system installation on NRAO Linux machines.
\item {\tt noScrat} \\
A list of AIPS disks to avoid for scratch files, max. 10.
\item {\tt nThreads} \\
The maximum number of threads allowed to be used.
This generally should not be more than the number of cores available.
\item {\tt disk} \\
The AIPS disk number to use for temporary storage of the data and images.
\end{itemize}

An example AIPSSetup.py file follows, items which may need to be
modified are marked by {\tt <====}.:
\begin{verbatim}
#   <====  Define AIPS  and FITS disks
adirs = [(None, "/export/data_1/GOLLUM_1"),
         (None, "/export/data_1/GOLLUM_2"),
         (None, "/export/data_1/GOLLUM_3"),
         (None, "/export/data_1/GOLLUM_4"),
         (None, "/export/data_2/GOLLUM_5"),
         (None, "/export/data_2/GOLLUM_6"),
         (None, "/export/data_2/GOLLUM_7"),
         (None, "/export/data_2/GOLLUM_8")]
fdirs = [(None, "/export/users/aips/FITS")]

############################# Initialize OBIT #####################
err     = OErr.OErr()
user    = 104                      # <==== set user number
ObitSys = OSystem.OSystem ("Script", 1, user, 0, [" "], \
                         0, [" "], True, False, err)
OErr.printErrMsg(err, "Error with Obit startup")
# Setup AIPS
AIPS.userno = user
AIPS_ROOT    = "/home/AIPS/"        # <==== set root of AIPS
AIPS_VERSION = "31DEC15/"           # <==== set AIPS version
DA00         = "/home/AIPS/DA00/"   # <==== set AIPS DA00 directory
#  <====  Define OBIT_EXEC for access to Obit Software 
OBIT_EXEC    = "/export/data_1/obit/ObitInstall/ObitSystem/Obit/"
# setup environment
ObitTalkUtil.SetEnviron(AIPS_ROOT=AIPS_ROOT, AIPS_VERSION=AIPS_VERSION, \
                        OBIT_EXEC=OBIT_EXEC, DA00=DA00, ARCH="LINUX", \
                        aipsdirs=adirs, fitsdirs=fdirs)
# List directories
ObitTalkUtil.ListAIPSDirs()
ObitTalkUtil.ListFITSDirs()
noScrat     = []         # <==== AIPS disks to avoid 
nThreads    = 6          # <====  Number of threads allowed
disk        = 1          # <====  AIPS disk number
\end{verbatim}

\section{The Process Overview}

The scripted processing uses the following processes.
Many of the default processing parameters are frequency dependent and
may be overridden and the various steps may be turned on or off.

The general approach to calibration and editing is to first apply
editing steps which can be applied to uncalibrated data to remove the
most seriously disturbed data.
Then an initial pass at calibration is done and a pass at the editing
needing calibrated data.
Calibration aids in the editing as calibrator data with no detections
are effectively removed and calibration with deviant amplitude
solutions are also removed.

As part of the calibration process, calibrators are self calibrated
which allows resolved sources, especially solar system objects to be
used. 
Diagnostic plots at various stages of the processing are generated.
These  include plots of calibration results as well as sample spectra.

The calibrated target data are then imaged, using wide-band imaging if
appropriate. 
Images, calibrated data and calibration tables are saved to FITS files
and a number of source dependent diagnostic plots are generated.
Finally, an HTML report is generated allowing easy examination and
access to the various products.
A processing log is kept containing most details of the processing.

Following is a summary of the processing.
Details and parameters which may be modified are described in a
section \ref{Details}.
Each of these steps if controlled by a switch which may be turned on and off.

\begin{enumerate}
\item Generation of parameter scripts from ASDM
\item Data converted to AIPS format
\item Hanning if necessary
\item Clear previous calibration
\item Copy initial FG table 
\item Flag end channels 
\item Apply Special Editing
\item Quack
\item Shadow Flagging 
\item Apply online (Tsys) calibration
\item Initial Time domain flagging
\item Initial RMS flagging of calibrators
\item Find reference antenna if not specified
\item Plot raw sample spectra 
\item Delay calibration
\item Bandpass calibration
\item Set X/Y gains and initial calibration
\item Self calibrate calibrators to get model
\item Phase calibration using calibrators
\item Amplitude calibration using calibrators
\item Flagging of calibrated data
\item Apply calibration and average data
\item Cross Pol clipping if XY, YX present in data
\item X-Y delay calibration if XY, YX present in data
\item Instrumental polarization calibration if XY, YX present in data
\item Plot final calibrated spectra 
\item Image targets
\item Generate source report
\item Save images, calibrated data
\item Contour plots of images
\item source UV diagnostic plots
\item Generate HTML summary
\item Cleanup AIPS directories
\end{enumerate}

Each of these steps if controlled by a switch which may be turned on
and off; the following appear at the bottom of the parameter file.
\begin{verbatim}
# Control, mark items as F to disable
T   = True
F   = False
check                  = parms["check"]      # Only check script, don't execute tasks
debug                  = parms["debug"]      # run tasks debug
parms["doLoadArchive"] = T                   # Load from archive?
parms["doHann"]        = parms["doHann"]     # Apply Hanning?
parms["doClearTab"]    = T                   # Clear cal/edit tables
parms["doCopyFG"]      = T                   # Copy FG 1 to FG 2
parms["doEditList"]    = parms["doEditList"] # Edit using editList?
parms["doQuack"]       = T                   # Quack data?
parms["doShad"]        = parms["doShad"]     # Flag shadowed data?
parms["doOnlineCal"]   = T                   # Apply online calibration
parms["doMedn"]        = T                   # Median editing?
parms["doRMSAvg"]      = T                   # Do RMS/Mean editing for calibrators
parms["doDelayCal"]    = T                   # Group Delay calibration?
parms["doBPCal"]       = T                   # Determine Bandpass calibration
parms["doXYFixGain"]   = T                   # set X/Y gains and initial calibration
parms["doImgCal"]      = T                   # Self calibrate calibrators
parms["doPhaseCal"]    = T                   # Phase calibration
parms["doAmpPhaseCal"] = T                   # Amplitude/phase calibration
parms["doAutoFlag"]    = T                   # Autoflag editing after final calibration?
parms["doCalAvg"]      = T                   # calibrate and average data
parms["doXYDelay"]     = T                   # X/Y Delay calibration
parms["doPolCal"]      = parms["doPolCal"]   # Instrumental polarization calibration
parms["doImage"]       = T                   # Image targets
parms["doSaveImg"]     = T                   # Save results to FITS
parms["doSaveUV"]      = T                   # Save calibrated UV data to FITS
parms["doSaveTab"]     = T                   # Save UV tables to FITS
parms["doKntrPlots"]   = T                   # Contour plots
parms["doDiagPlots"]   = T                   # Source diagnostic plots
parms["doMetadata"]    = T                   # Generate metadata dictionaries
parms["doHTML"]        = T                   # Generate HTML report
parms["doVOTable"]     = T                   # VOTable report
parms["doCleanup"]     = T                   # Destroy AIPS files

# diagnostics
parms["doSNPlot"]      = T                   # Plot SN tables etc
parms["doBPPlot"]      = T                   # Plot BP tables etc
parms["doReport"]      = T                   # Individual source report
parms["doRawSpecPlot"] = @PLOTSRC@!='None'   # Plot Raw spectrum
parms["doSpecPlot"]    = @PLOTSRC@!='None'   # Plot spectrum at various stages
\end{verbatim}

\section{Tuning parameters}\label{Tune}
The ASDM includes intent information for sources and scans which are
used by the ALMACal.ALMAPrepare to create initial parameter files;
these may not be adequate for all purposes.
The following discuss some parameters which may need adjustment.
Several of these are marked as
\begin{verbatim}
parms['some_parameter'] = value   # ******************* Set this ****
\end{verbatim}
in the derived parameter files.
\begin{enumerate}
\item doCleanup\\
The default behavior of the script is to delete the AIPS data files on
successful completion.
If you want to try several strategies or do further processing on the
AIPS data, turn this to False (or F)
\item CalModelFlux\\
The ``CalModelFlux'' member of the calibrator source structure can be
used to specify the flux density of a calibrator source, e.g.
\begin{verbatim}
ACals[0]['CalModelFlux'] = 1.558   # ALMA 2013-12-21 1.558 +/- 0.06
\end{verbatim}
ALMA calibrator monitoring results are available at
https://almascience.eso.org/sc/.
Setting a model may be most useful for ACals, PCals and BPCals.
A more general discussion of specifying calibrator information see
Sect.  \ref{calmodel}.
\item IClip \\
Absolute clipping level for any calibrator/target is given by
IClip=[level\_Jy,0.1].
This can be used to flag wild values
\item CAchAvg \\
The number of channels to average when calibrating/averaging the data.
This is useful for continuum observations or to reduce the spectral
resolution of spectroscopic data.
\item CalAvgTime \\
The time in minutes to average data  when calibrating/averaging the data.
\item plotSource \\
This is the name of the source to to be used for diagnostic bandpass
plots, see Sect. \ref{Details} item \ref{rawplot}.
This defaults to the first scan on the first bandpass calibrator named.
\item plotTime \\
Time range (days) for which to use plotSource.
\item XYGainSource \\
This is the name of the source to set the X/Y gains (see
Sect. \ref{Details} item \ref{xygain}).
This defaults to the first scan on the first bandpass calibrator named.
\item XYGainTime \\
Time range (days) for which to use XYGainSource.
\item XYDelaySource \\
If the data is to be polarization calibrated, this item is the name of
the polarized calibrator to be used to derive the cross--polarized
delay.
This is not specified in the ASDM and if needed must be manually added.
\item XYDelayTime \\
If the data is to be polarization calibrated, this item is the
time-range (days) of the data on XYDelaySource to be used.
\item PClist \\
This is a list of sources to use in the instrumental polarization.
The ASDM has a rather restricted view of this and more than one
sources can be used (up to 10).
See Sect. \ref{Details} item \ref{polcal} for a description of how to specify models per
calibrator in PCCalPoln. 
\end{enumerate}


\section{Script Stage Details}\label{Details}
Details of the various processing steps and the parameters are
described in the following.
Processing parameters are stored in a python dict named parms and may
be specified in the parameter script as
\begin{verbatim}
parms["parameter"] = value
\end{verbatim}
Tables in the following give the parameter name, default value and a
description.
%Note: Not all stages are performed by default.

\begin{enumerate}
\item Generation of parameter scripts from ASDM\\
This step is performed from ObitTalk to generate the parameter
script(s).
Multiple, compatible data-sets may be concatenated by giving a list of
the base directories of the ASDM/BDFs.
\begin{verbatim}
>>> import ALMACal
>>> ASDMRoots = ["/export/myData/12A-999.sb9332941.eb9360588.../",]
>>> ALMACal.ALMAPrepare(ASDMRoots, err, project="myData")
\end{verbatim}
This will parse the ASDMs indicated and generate a parameter script for
each configuration/number of channel combinations needed to process
all data.
Note: this will also include configurations used for calibration
purposes only, such as offset pointing, so use Obit task ASDMList to see
which configurations have useful data.
For each configuration/number of channel, a parameter script with name
of the form ALMAContParm\_$<$project$>$\_Cfg$<$config$>$\_Nch$<$no channels$>$.py
will be generated.

\begin{center}
\begin{tabular}{|l|c|l|}
\hline
ASDMRoots & & Root directories of ASDM/BDF data\\
project  & & Project name, 12 or fewer characters, \\
 & & used as AIPS file name \\
template & ALMATemplateParm.py & name of the parameter template
file\\
parmFile &  & Name of desired parameter file, \\
 & & generated if not given\\
\hline
\end{tabular}
\end{center}
%
\item General control parameters\\
These parameters control the naming of files, whether UV data is used
in compressed form and script debugging control.
\begin{center}
\begin{tabular}{|l|c|l|}
\hline
 project & ?? &  Project name, 12 or fewer characters,  used as AIPS file name\\
 session & ?? &  session code, generated from configuration and no. channels\\
 band  & ?? &  Observing band code, derived from ASDM frequencies\\
 Compress & False &  Use compressed (scaled 16 bit integers) UV data?\\
 check & False &  Only check script, don't execute tasks\\
 debug & False &  run tasks debug\\
  &  &  \\
\hline
\end{tabular}
\end{center}
%
\item Data converted to AIPS format\\
The bulk of the processing uses AIPS format UV data and images.
The ASDM/BDF data is first converted to an AIPS data file using Obit/BDFIn.
The details of the AIPS configuration are given in the AIPSSetup.py
file provided to the processing script.
\begin{center}
\begin{tabular}{|l|c|l|}
\hline
doLoadArchive  & True & Load AIPS data from archive? \\
archRoots  & ?? &  User specified list of ASDMs/BDFs\\
selConfig  & ?? &  Frequency configuration, generated from ASDM\\
seq     & 1 &  AIPS sequence number to use \\
selBand & ??  &  Data band-code, derived from ASDM\\
selChan & ??  &  Number of spectral channels, derived from ASDM\\
selNIF  & ??  &  Number of spectral windows (IFs), derived from ASDM\\
calInt  & ??  &  Calibration table interval (min), ALMA config. dependent\\
\hline
\end{tabular}
\end{center}
%
\item Hanning \\
Very strong, narrow signals will produce ``Gibbs'' ringing
due to the truncation of the lag spectra.
Hanning smoothing can be used to suppress this effect.
\begin{center}
\begin{tabular}{|l|c|l|}
\hline
doHann  & False &  Apply Hanning smoothing? \\
doDescm & True  & If True, drop every other channel after smoothing  \\
\hline
\end{tabular}
\end{center}
%
\item Clear previous calibration\\
If the script is restarted it is frequently desirable that previous
attempts at calibration and editing be removed.
\begin{center}
\begin{tabular}{|l|c|l|}
\hline
doClearTab   & True & Clear cal/edit tables? \\
doClearGain  & True & Clear SN and CL tables $>$ 1? \\
doClearFlag  & True & Clear FG tables $>$ 1? \\
doClearBP    & True & Clear BP tables? \\
\hline
\end{tabular}
\end{center}
%
\item Copy initial FG table \\
To allow restarting of the flagging, the on-line flags which are in FG
table 1 are copied to table 2 and new flags added there.
This should be turned off if the script is restarted except at the
beginning.
Note, this only affects the ASDM flags, binary flags are applied
directly to the data weights.
\begin{center}
\begin{tabular}{|l|c|l|}
\hline
doCopyFG  & True & Copy FG 1 to FG 2? \\
\hline
\end{tabular}
\end{center}
%
\item Flag end channels\\
The first and last few channels in each IF (SW) are flagged if FG table 2.
This can be turned off by setting BChDrop and EChDrop to 0.
\begin{center}
\begin{tabular}{|l|c|l|}
\hline
BChDrop  & min (32,max(2, (nchan/64)) & \# of chan. to flag at the beginning \\
EChDrop  & min (32,max(2, (nchan/64)) & \# of chan. to flag at the end \\
\hline
\end{tabular}
\end{center}
%
%
\item Apply Special Editing\\
If some data are known to be bad, e.g. no receiver, then this
information can be passed to the script.
if doEditList is True, each entry is a python dict with the following:
\begin{itemize}
\item{\tt timer:} The affected time range as a pair of strings of the
form day/hour:min:sec.
\item{\tt Ant:}A baseline specification as a pair of antenna numbers,
if the second is zero, then all baselines to the first antenna number
is flagged.  
If the first is also zero, then all antennas are flagged
\item{\tt IFs:} Range (1-rel)  of IFs (spectral windows) to flag.
If the second is zero then all IFs higher than the first are flagged.
\item{\tt Chans:} Range (1-rel) of channels to flag.
If the second is zero then all channels higher than the first are flagged.
\item{\tt Stokes:} Array of flags, 1$=>$flag, $0=>$ not flag; in order
  XX, YY, XY, YX.
\item{\tt "Reason:} Up to 24 characters giving reason.
\end{itemize}
an example:
\begin{verbatim}
parms["doEditList"]  = True        # Edit using editList?
parms["editList"] = [
    {"timer":("0/00:00:0.0","5/00:00:0.0"),"Ant":[1,0],
     "IFs":[1,0],"Chans":[1,0],  "Stokes":'1111',"Reason":"No Rcvr"}
]
\end{verbatim}

\begin{center}
\begin{tabular}{|l|c|l|}
\hline
doEditList  & False & Edit using editList? \\
editFG      & 2     & Table to apply edit list to\\
 editList   & [ ]   & List of data to flag \\
  &  &  \\
\hline
\end{tabular}
\end{center}
%
\item Quack\\
Data at the beginning and end of each scan can be flagged using
Obit/Quack. 
\begin{center}
\begin{tabular}{|l|c|l|}
\hline
doQuack       & True    & Quack data? \\
quackBegDrop  & 0.05    & Time to drop from start of each scan (min) \\
quackEndDrop  & 0.0     & Time to drop from end of each scan (min) \\
quackReason   & "Quack" & Reason string \\
\hline
\end{tabular}
\end{center}
%
\item Shadow Flagging \\
In the more compact ALMA configurations, some antennas may shadow
others at times.
The affected data may be flagged using task UVFlag.
\begin{center}
\begin{tabular}{|l|c|l|}
\hline
doShad  & False & Do shadow flagging?\\
shadBl  & 12.0 &  Minimum shadowing baseline (m)\\
\hline
\end{tabular}
\end{center}
%
\item Apply online (Tsys) calibration \\
Online Tsys measurements are extracted from the ASDM when the data is
filled into AIPS format and written in the form of an AIPS SN version
1 table.
This table can be applied to the initial Calibration (AIPS CL) table
to produce AIPS CL table version 2.
\begin{center}
\begin{tabular}{|l|c|l|}
\hline
doOnlineCal  & True & Apply TSys calibration?\\
\hline
\end{tabular}
\end{center}
%
\item Initial Time domain flagging\\
Obit task MednFlag can be used to flag data by amplitudes deviant from
a running median by more than a specified amount.
This is performed independently on each data stream (baseline,
channel, IF, poln).
At this point the data are uncalibrated.
\begin{center}
\begin{tabular}{|l|c|l|}
\hline
doMedn       & True & Median editing? \\
mednSigma    & 10.0 &  Sigma clipping level\\
mednTimeWind & 1.0  &  Window width (min) for median flagging\\
mednAvgTime  & 10.0/60. & Averaging time (min) \\
mednAvgFreq  & 0    &  1$=>$avg mednChAvg chans, 2$=>$avg all chan,\\
 & & 3$=>$avg chan and IFs\\
mednChAvg    & 1    &  Number of channels to average\\
\hline
\end{tabular}
\end{center}
%
\item Initial RMS flagging of calibrators \\
Calibrators are expected to be simple and have  significant SNR so can
be edited by having an RMS/average amplitude of less than some
amount.
Discrepant calibrator data can be flagged in this step using Obit task
AutoFlag.
\begin{center}
\begin{tabular}{|l|c|l|}
\hline
doRMSAvg   & True &  Edit calibrators by RMS/Avg?\\
RMSAvg     & 3.0  &  Max RMS/Avg for time domain RMS filtering\\
RMSTimeAvg & 1.0  &  Time averaging (min)\\
\hline
\end{tabular}
\end{center}
%
\item Reference antenna\\
The choice of reference antenna is of some importance but nothing in
the ASDM helps in this choice.
If the reference antenna (parms[``refAnt'']) is unspecified (0),
this step runs Obit task Calib on the bandpass calibrator(s) (assumed to
give good fringes).
The resultant SN table is then examined for the antenna with the
maximal amount of valid solutions and with the highest average SNR;
this antenna is used as the reference antenna.
Values found from a previous run will be stored in a python pickle file.
\begin{center}
\begin{tabular}{|l|c|l|}
\hline
refAnt  &  & Reference antenna, if $<=$ 0 then determine \\
 BPCals &  & Determined from the ASDM \\
 bpsolint1 &  &  Bandpass first solution interval, \\
\hline
\end{tabular}
\end{center}
%
\item Plot Raw spectra \label{rawplot}\\
At this point, plots of sample spectra can be made to display
problematic data; baselines to refAnt for the specified 
source and timerange are plotted for the parallel correlations.
\begin{center}
\begin{tabular}{|l|c|l|}
\hline
doRawSpecPlot  & True &  Plot diagnostic raw spectra?\\
plotSource     & & Default is first bandpass calibrator\\
plotTime       & & List of start and end time in days.\\
  &  & Default is first bandpass calibrator scan\\
refAnt  &  & Reference ant., baselines to refAnt are plotted \\
\hline
\end{tabular}
\end{center}
\newpage
%
\item Delay calibration\\
Parallel hand group delays are solved for using the list of
calibrator models in DCals.
Obit task Calib solves for the delays which are then smoothed and
applied to all sources in a new CL table using Obit task CLCal.
Solutions can be plotted as well as sample spectra applying the delay
calibration.
For ALMA data the delay residuals are generally very small.
\begin{center}
\begin{tabular}{|l|c|l|}
\hline
doDelayCal  & True &  Determine/apply delays? \\
DCals  &  & The list of delay calibrators are determined from the ASDM, \\
  &  & all amplitude, phase and bandpass calibrator. \\
  &  & The list of models is determined from the parameter script \\
  & & using standard calibrator models. \\
delayBChan  &  &  first channel to use in delay solutions \\
  &  &  max(2, 0.05*nchan)\\
delayEChan  &  &  highest channel to use in delay solutions\\
  &  &  min(nchan-2, nchan-0.05*nchan\\
solInt  &  &  Solution interval (min), config. dependent\\
  &  &  A:2 sec, B: 5 sec, C:10 sec, D:15 sec.\\
refAnts  & [refAnt] & Delay reference ant., baselines to refAnt are plotted \\
doTwo  & True &  Use two baseline combinations in delay cal\\
delayZeroPhs &  True &  Zero phase in Delay solutions?\\
doSNPlot       & True &  Plot calibration solutions?\\
doSpecPlot     & True &  Plot diagnostic calibrated spectra?\\
plotSource     & & Source to plot spectra.\\
plotTime       & & List of start and end time in days.\\
\hline
\end{tabular}
\end{center}
\newpage
%
\item Bandpass calibration\\
Bandpass calibration uses Obit task BPass and calibrator model list BPCals.
BPass does a two pass calibration, the first doing a phase only
calibration to straighten out the phases followed by a longer amplitude
and phase calibration using blocks of channels.
The resultant solutions are then combined into a BP table
\begin{center}
\begin{tabular}{|l|c|l|}
\hline
doBPCal  & True &  Determine/apply bandpass calibration? \\
BPCals  &  & The list of bandpass calibrators is determined from the ASDM, \\
  &  & The list of models is determined from the parameter script \\
  & & using standard calibrator models. \\
bpsolint1  &  &  BPass phase correction solution \\
bpsolint2  & 10.0 & BPass bandpass solution interval (min) \\
bpsolMode  & 'A\&P'  &  Bandpass type 'A\&P', 'P', 'P!A'\\
bpBChan1  & 1 &  Low freq. channel,  initial cal\\
bpEChan1  & 0 &  Highest freq channel, initial cal, 0$=>$all\\
bpBChan2  & 1 &  Low freq. channel for BP cal\\
bpEChan2  & 0 &  Highest freq channel for BP cal,  0$=>$all \\
bpChWid2  & 3 &  Number of channels in running mean BP soln\\
bpDoCenter1  & None &  Fraction of  channels in 1st, overrides
bpBChan1, bpEChan1\\
bpUVRange  & [0.0,0.0] &  UV range for bandpass cal zeroes$=>$ all\\
refAnt  &  & BP reference ant., baselines to refAnt are plotted \\
doSpecPlot     & True &  Plot diagnostic calibrated spectra?\\
plotSource     & & Source to plot spectra.\\
plotTime       & & List of start and end time in days.\\
\hline
\end{tabular}
\end{center}
\newpage
%
\item Fix X/Y gain ratio and initial calibration \label{xygain} \\
ALMA uses linearly polarized feeds and even parallel correlations
respond to a combination of Stokes I, Q and U.
This is a combination of I and plus or minus a function of Q and U for
the XX and YY correlations.
As the calibrators, generally QSOs, have significant linear
polarization, gains derived from the parallel polarized will be in
error and differently in error for the XX and YY correlations.
The average of XX and YY will, to first order, give Stokes I but if
polarimetric calibration is needed, this effect must be corrected.
This step is to fix the X/Y gain ratio to the value for a specific
calibrator for a specific time.
If the calibrator is unpolarized, the gain ratio will be 1.0 at all
times and all data can be used.
For a polarized source, the time range should be restricted to a range
over which the parallactic angle has a small change.
Calibration after this step prior to instrumental polarization
calibration will average XX and YY and will not modify the X/Y gain
ratio.
The instrumental polarization calibration can include determination of
the X/Y gain ratio.
\begin{center}
\begin{tabular}{|l|c|l|}
\hline
doXYFixGain   & True &  Fix X/Y gain ratio\\
XYGainSource  &  & Calibrator source to be used \\
XYGainTime &  & start and end times for calibration in days \\
refAnt     &  & Reference anttenna\\
\hline
\end{tabular}
\end{center}
%
\item Self calibrate calibrators \\
Calibrators, especially solar system objects, may be resolved.
Self calibration is used to derive a source model to be used is
subsequent amplitude and phase calibration.
\begin{center}
\begin{tabular}{|l|c|l|}
\hline
doImgCal  & True &  Image/self calibrate calibrators\\
CalFOV    & ??   & Field of view (deg) to image \\
          &      & Frequency and maximum baseline dependent.\\
outCClass & `ISfCal' &  Image AIPS class \\
refAnt      &  & Reference antenna. \\
maxPSCLoop  & 1        & Max. number of phase self cal loops\\
minFluxPSC  & 0.05     & Min flux density peak for phase self cal\\
solPInt     &   ??     & Phase self cal solution interval (min), $\nu$ dependent \\
maxASCLoop  & 1        & Max. number of Amp+phase self cal loops\\
minFluxASC  & 0.5      & Min flux density peak for amp+phase self cal\\
solAInt     &    3     & amp+phase self cal solution interval (min), \\
avgPol      & True     & Average poln in self cal?\\
avgIF       & False    & Average IF in self cal?\\
minSNR      & 4.0      & Minimum Allowed SNR in self cal\\
\hline
\end{tabular}
\end{center}
%
\item Phase Calibration \\
An initial, short term phase calibration is done on the calibrators to
remove the phase variations.
This uses the calibrator models derived from the self calibration
stage.
Both ``phase'' and ``amplitude'' calibrators are used.
Solutions are applied to the previous CL table to create a new
CL table.
\begin{center}
\begin{tabular}{|l|c|l|}
\hline
doPhaseCal  & True & Do phase calibration? \\
PCals  &  & The list of phase calibrator models \\
ACals  &  & The list of amplitude calibrator models \\
refAnt  &  & Reference antenna \\
solPInt &  &  Solution interval (min), $\nu$ dependent:\\
ampBChan  & max(2, 0.05*nchan) & first channel to use in solutions \\
ampEChan  &  min(nchan-2, &  highest channel to use in solutions\\
          &  nchan-0.05*nchan) &  \\
ampScalar  & False &  Ampscalar solutions?\\
\hline
\end{tabular}
\end{center}
%
\item Amp \& phase Calibration \\
Standard flux density calibrators have their flux densities entered
into the SU table using Obit task SetJy, other calibrators have their
flux density entries set to the value of CalModelFlux, if given, else 1.0.
All the amplitude and phase calibrators have Obit/Calib run using
their models and doing amplitude and phase solutions.
Solutions are then median window smoothed using Obit/SNSmo to time
solSmo clipping really wild points.
Obit task GetJy then solves for the flux densities for non flux
density calibrators and adjusts the SU and SN tables. If doAmpEdit is
True, solutions in each IF (spectral window) more than ampSigma from
the mean are flagged both in the SN table and in FG table ampEditFG.
Finally solutions are applied to the previous CL table to create a new
CL table.
Solution plots are written into file\\
parms["project"]+"\_"+parms["session"]+"\_"+parms["band"]+"APCal.ps".
\begin{center}
\begin{tabular}{|l|c|l|}
\hline
doAmpPhaseCal  & True & Do amplitude and phase calibration? \\
ACals  &  & The list of amplitude calibrators are determined from the ASDM, \\
  &  & The list of models is determined from the parameter script \\
  & & using standard calibrator models. \\
PCals  &  &  The list of phase calibrators are determined from the ASDM\\
refAnt  &  & Reference antenna \\
solInt  &  &  Solution interval (min), config. dependent:\\
ampBChan  &  & first channel to use in A\&P solutions \\
  &  &   max(2, 0.05*nchan)\\
ampEChan  &  &  highest channel to use in A\&P solutions\\
  &  &  min(nchan-2, nchan-0.05*nchan)\\
solSmo  & 0.0 &  Smoothing interval for Amps (min)\\
ampScalar  & False &  Ampscalar solutions?\\
doAmpEdit  & True  & Edit/flag on the basis of amplitude solutions \\
ampSigma  & 20.0 &  Multiple of median RMS about median gain to clip/flag\\
ampEditFG  & 2 &  FG table for editing \\
doSNPlot       & True &  Plot calibration solutions?\\
\hline
\end{tabular}
\end{center}
%
\item Flagging of calibrated data\\
Calibrated data are then edited using Obit/AutoFlag. 
Data with amplitudes outside of a given range are flagged and data
overly discrepant from a running median in frequency is flagged.
\begin{center}
\begin{tabular}{|l|c|l|}
\hline
doAutoFlag  & True &  Autoflag editing after first pass calibration?\\
IClip     & [200.,0.1] &  AutoFlag Stokes I clipping\\
minAmp    & 1.0e-5 & Minimum allowable amplitude \\
timeAvg   & 0.33 &  AutoFlag time averaging in min.\\
doAFFD    & False & do AutoFlag frequency domain flag \\
FDmaxAmp  & IClip[0] &  Maximum average amplitude (Jy)\\
FDmaxV    & VClip[0] &  Maximum average VPol amp (Jy)\\
FDwidMW   & 31 & Width of the median window \\
FDmaxRMS  & [5.0,.1] &  FDmaxRMS\\
FDmaxRes  & 6.0  &  Max. residual flux in sigma\\
FDmaxResBL & 6.0 &  Max. baseline residual\\
FDbaseSel  & [0,0,0,0] & Channels for baseline fit \\
\hline
\end{tabular}
\end{center}
%
\item Calibrate and average data\\
The calibration and editing files are then applied with possible
averaging in time and/or frequency.
This uses Obit/Splat which writes a multi-source file.
\begin{center}
\begin{tabular}{|l|c|l|}
\hline
doCalAvg  & True &  Calibrate and average?\\
avgClass  & "UVAv?" & AIPS class of calibrated/averaged UV data \\
seq  & 1 &  AIPS sequence \\
CalAvgTime  &  &  Time for averaging calibrated UV data (min) \\
avgFreq  & 0 &  $0=>$ no averaging, $1=>$avg chAvg chans,\\
 & &$2=>$avg all, $3=>$avg chan and IFs\\
chAvg  & 1 & Number of channels to average \\
CABChan& 1 &  First channel to copy \\
CAEChan& 0 &  Highest channel to copy, $0=>$ all higher than CABChan\\
CABIF  & 1 &  First IF to copy \\
CAEIF  & 0 &  Highest IF to copy, $0=>$ all higher than CABIF\\
Compress & False &  Write compressed UV data?\\
\hline
\end{tabular}
\end{center}
%
\item Cross Pol clipping\\
if XClip[0] is not None,  cross polarized data with amplitudes $>$ XClip[0] are flagged.
\begin{center}
\begin{tabular}{|l|c|l|}
\hline
XClip  & [5.0,0.05] & AutoFlag cross-pol clipping, None$=>$ no flagging \\
\hline
\end{tabular}
\end{center}
%
\item X-Y  delay calibration\\
If the data contain XY and YX correlations and polarimetric results
are desired, the delay difference between the X and Y systems is needed.
This step determines the x-y delay from a single calibrator.
These calibrators are not specified in the ASDM and will need to be
inserted manually as parms['XYDelaySource'] and  parms['XYDelayTime'].
If the first source in this list is not None, and the data contains
the XY and YX correlations, the x-Y delay calibration is performed. \\
%This is not yet supported in the automated scripting.
\begin{center}
\begin{tabular}{|l|c|l|}
\hline
doXYDelay   & ?? &  Do X-Y delay calibration?\\
XYDelaySource & ?? &  Calibrator, skip if None \\
XYDelayTime   & ?? &  timerange for calibration \\
xyBChan     & 1            & First (1-rel) channel number\\
xyEChan     & 0            & Highest channel number. $0=>$ high in data. \\
xyUVRange   &  [0.0,0.0]   & Range of baseline used in kilo wavelengths, zeros=all\\
xyDoCal     & 2            & Apply calibration table? positive$=>$calibrate\\
xygainUse   & 0            & CL/SN table to apply, $0=>$highest\\
%rlDoBand    & 1            & If $> 0$ apply bandpass calibration \\
%rlBPVer     & 0            & BP table to apply, $0=>$highest\\
xynumIFs    & 1            & Number of IFs per solution \\
xyflagVer   & 2            & FG table version to apply \\
refAnt      &              & Reference antenna \\
\hline
\end{tabular}
\end{center}
%
\item Instrumental polarization calibration \label{polcal} \\
Determine instrumental polarization from a list of calibrators if the data
contains XY and YX correlations.
The function ALMACal.ALMAPrepare sets this list to those with an
intent ``CALIBRATE\_POLARIZATION''.
Parameter parms["doPolCal"] is set True if PCInsCals is not empty and
the calibration performed if the data contains the XY and YX
correlations. 
Calibration uses Obit task PCal which determines antenna and source
polarization parameters on blocks of channels in a running window.
The antenna parameters are the ellipticity and orientation of the
feed; see Obit Development Memo 32 (in preparation) for details.\\
Model parameters can be specified in PCCalPoln as a list of tuples
corresponding to sources in PCInsCals, each tuple is:
\begin{enumerate}
\item PPol\\
Fractional poln, $<$0 $=>$ fit
\item EVPA\\
Polarization angle at reference frequency in deg
\item RM\\
 Rotation measure (rad/m**2)
\end{enumerate}
\begin{center}
\begin{tabular}{|l|c|l|}
\hline
doPolCal  & ??     &  Determine instrumental polarization? \\
PCInsCals & ??     &  Instrumental poln cals, name or list of names\\
PCCalPoln &  None  &  Models for PCInsCals, None$=>$fit all \\
PCSolInt  & 2.0    &  Instrumental solution interval (min), \\
          &        & $0=>$ scan average(?) \\
PCRefAnt  & 0      &  Reference antenna, defaults to refAnt\\
PCSolType & "    " &  Solution type, "LM  " (better), "    " (faster)\\
PCChInc   & 5      &  Channel step in spectrum \\
PCChWid   & 5      &  Number of channels to average \\
doFitOri  &  False &  Fit (linear feed) orientations? \\
doFitXY   &  True  &  Fit X-Y gain phase\\
doPol     & False  &  Apply polarization cal in subsequent calibration?\\
PDVer     & 1      &  Apply PD table in subsequent polarization cal?\\
\hline
\end{tabular}
\end{center}
%
\item Plot final calibrated spectra\\
At this point, plots of sample spectra can be made to display
calibrated data.
\begin{center}
\begin{tabular}{|l|c|l|}
\hline
doSpecPlot     & True &  Plot diagnostic spectra?\\
plotSource     &      & Source to plot\\
plotTime       &      & List of start and end time in days.\\
refAnt         &      & Reference ant., baselines to refAnt are plotted \\
\hline
\end{tabular}
\end{center}
%
\item Image targets \\
All targets are imaged and deconvolved using Obit/Imager or
Obit/MFImage if wideband imaging needed (fractional spanned bandwidth
$\ge$ MBmaxFBW).
Phase only and amp and phase self calibration may be applied if sources
exceed given thresholds.
If wideband imaging is used, then the resultant images are cubes
having planes:
\begin{enumerate}
\item Total intensity at reference frequency.
\item Spectral index at reference frequency
\item any higher order planes
\item One plane for each of the coarse frequency samples.
\end{enumerate}
\begin{center}
\begin{tabular}{|l|c|l|}
\hline
doImage     & True     & Image targets? \\
targets     & [?]       & Target list set from ASDM, empty$=>$all\\
seq         & 1        & AIPS sequence for images \\
doPol       & True     & Apply polarization cal?\\
PDVer       &  1       & Apply PD table?\\
outIclass   & "IClean" & Image AIPS class\\
Stokes      & "I"      & Stokes to image \\
Robust      & 0.0      & Weighting robust parameter\\
FOV         &          & Field of view radius in deg, average $\nu$ dependent:\\
Niter       & 500      & Max number of CLEAN iterations\\
minFlux     & 0.0      & Minimum CLEAN flux density (Jy) \\
minSNR      & 4.0      & Minimum Allowed SNR in self cal\\
maxPSCLoop  & 1        & Max. number of phase self cal loops\\
minFluxPSC  & 0.05     & Min flux density peak for phase self cal\\
solPInt     &          & Phase self cal solution interval (min), $\nu$ dependent \\
            &          &       $\nu<1$ GHz, L,C,X,Ku,K,Ka: 0.25, Q band:0.10\\
solPMode    & "P"      & Solution mode for phase self cal\\
solPType"   & "L1"     & Solution type for phase self cal\\\
maxASCLoop  & 1        & Max. number of Amp+phase self cal loops\\
minFluxASC  & 0.5      & Min flux density peak for amp+phase self cal\\
solAInt     &   3      & amp+phase self cal solution interval (min), $\nu$ dependent\\
solAMode    & "A\&P"   & Amp and phase self cal \\
solAType    & "L1"     & Solution type for Amp and phase self cal\\
avgPol      & True     & Average poln in self cal?\\
avgIF       & False    & Average IF in self cal?\\
nTaper      & 0        & Number of additional imaging multi-resolution tapers\\
Tapers      & [20.0,0.0] &  List of tapers in pixels\\
do3D        & False    & Make ref. pixel tangent to celest. sphere for each facet\\
noNeg       & False    & Allow negative components in self cal model?\\
BLFact      & 1.01     & Baseline dependent time averaging for $>1.0$?\\
BLchAvg     & True     & Baseline dependent frequency averaging?\\
doMB        &  ??      & Set in parameter script depending on spanned bandwidth\\
MBnorder    & 1        & Order of wideband imaging \\
MBmaxFBW    & 0.05     & max. MB fractional bandwidth\\
CleanRad    & None     & CLEAN radius about center or None=autoWin\\
\hline
\end{tabular}
\end{center}
%
\newpage
\item Generate report\\
\begin{center}
\begin{tabular}{|l|c|l|}
\hline
doReport    & True     & Generate source report? \\
targets     & [?]      & Target list set from ASDM, empty$=>$all\\
seq         & 1        & AIPS sequence for images \\
outIclass   & "IClean" & Image AIPS class\\
Stokes      & "I"      & Stokes imaged \\
\hline
\end{tabular}
\end{center}
%
\item Save images, calibrated data\\
Images and calibrated/averaged data and calibration tables are written
to FITS files.
File names begin with \\
parms["project"]+parms["session"]+parms["band"]
followed by \\
$<$source\_name$>$+$<$Stokes$>$+"Clean.fits" for images and
"Cal.uvtab" for calibrated data and "CalTab.uvtab" for calibration
tables from the original data.
\begin{center}
\begin{tabular}{|l|c|l|}
\hline
doSaveImg & True &  Save target images to FITS?\\
targets   & [?]  & Target list set from ASDM, empty$=>$all\\
doSaveUV  & True & Save calibrated UV data for AIPS/FITAB format? \\
doSaveTab & True &  Save calibration tables for AIPS/FITAB format?\\
\hline
\end{tabular}
\end{center}
%
\item Contour plots of images\\
Contour plots are generated for target images.
Plot names are \\
parms["project"]+"\_"+parms["session"]+"\_"+parms["band"]
followed by the source name and ".cntr.ps" which are also converted
to jpeg with the suffix "jpg".
\begin{center}
\begin{tabular}{|l|c|l|}
\hline
doKntrPlots & True &  Generate contour plots?\\
targets   & [?]  & Target list set from ASDM, empty$=>$all\\
\hline
\end{tabular}
\end{center}
%
\item UV diagnostic plots\\
Plots of amplitude vs. baseline length, real vs. imaginary and UV
coverage are generated.
Plot names are \\
parms["project"]+"\_"+parms["session"]+"\_"+parms["band"]
followed by the source name and ".amp.ps",  ".ri.ps", or ".uv.ps"
which are also converted to jpeg with the suffix "jpg".
\begin{center}
\begin{tabular}{|l|c|l|}
\hline
doDiagPlots & True & Make UV diagnostic plots per source? \\
targets   & [?]   & Target list set from ASDM, empty$=>$all\\
\hline
\end{tabular}
\end{center}
%
\item Generate HTML Summary \\
Generate an HTML page with source statistics and links to the various
plots.
\begin{center}
\begin{tabular}{|l|c|l|}
\hline
doHTML  & True & Generate HTML reports? \\
\hline
\end{tabular}
\end{center}
%
\item Cleanup \\
AIPS data and image files are zapped.
\begin{center}
\begin{tabular}{|l|c|l|}
\hline
doCleanup  & True & Clean out AIPS directories? \\
\hline
\end{tabular}
\end{center}
%
\end{enumerate}

\section {The Products}\label{products}

\begin{itemize}
\item Calibrated (u,v) dataset with calibration and flagging tables in
AIPS FITAB format -- Tables from initial data and averaged
visibilities per input dataset. 
These files are\\
parms["project"]+parms["session"]+parms["band"]+"Cal.uvtab"
and parms["project"]+parms["session"]+parms["band"]+"CalTab.uvtab".
\item FITS Images -- for each target object in files\\
parms["project"]+"\_"+parms["session"]+"\_"+parms["band"]+\\
source\_name+".IClean.fits".\\
If wideband imaging is used, then the resultant images are cubes
having planes:
\begin{enumerate}
\item Total intensity at reference frequency.
\item Spectral index at reference frequency
\item any higher order planes
\item One plane for each of the coarse frequency samples.
\end{enumerate}
\item Diagnostic plots -- calibration and several per source.
The project plots have prefix
parms["project"]+"\_"+parms["session"]+"\_"+parms["band"] and are
\begin{itemize}
\item{\tt RawSpec.ps:} AIPS/POSSM plots of sample spectra with initial
editing but no calibration applied.
\item{\tt DelaySpec.ps:} AIPS/POSSM plots of sample spectra with initial
editing and delay calibration applied.
One set per pass through the calibration.
\item{\tt BPSpec.ps:} AIPS/POSSM plots of sample spectra with initial
editing and delay and bandpass calibration applied.
One set per pass through the calibration.
\item{\tt Spec.ps:} AIPS/POSSM plots of sample spectra with final
editing and calibration applied.
\item{\tt XYSpec2.ps:} AIPS/POSSM plots of sample XY and YX spectra
with final editing and calibration applied.
\item{\tt DelayCal.ps:} AIPS/SNPLT plots of delay calibration.
\item{\tt APCal.ps:} AIPS/SNPLT plots of amplitude and phase calibration.
\end{itemize}
The source plots have prefix
parms["project"]+"\_"+parms["session"]+"\_"+parms["band"] and are
\begin{itemize}
\item{\tt source\_name.cntr.jpg:} Source image contour plot as jpeg
\item{\tt source\_name.cntr.ps:} Source image contour plot as postscript
\item{\tt source\_name.amp.jpg:} Source amp. vs baseline plot as jpeg
\item{\tt source\_name.amp.pdf:} Source amp. vs baseline plot as pdf
\item{\tt source\_name.amp.ps:} Source amp. vs baseline plot as postscript
\item{\tt source\_name.ri.jpg:} Source real vs imaginary plot as jpeg
\item{\tt source\_name.ri.pdf:} Source real vs imaginary plot as pdf
\item{\tt source\_name.ri.ps:} Source real vs imaginary plot as postscript
\item{\tt source\_name.uv.jpg:} Source uv coverage plot as jpeg
\item{\tt source\_name.uv.pdf:} Source uv coverage plot as pdf
\item{\tt source\_name.uv.ps:} Source uv coverage plot as postscript
\end{itemize}
\item Reports and logs created during the process\\
The logfile is\\
parms["project"]+"\_"+parms["session"]+"\_"+parms["band"]+".log",
and the HTML report is\\
parms["project"]+"\_"+parms["session"]+"\_"+parms["band"]+".report.html".
%\item Meta-data for a VOTable to describe the products
\end{itemize}

The file set comprising all files and the meta-data are stored in a single
directory.  
%For approved pipeline use, this directory is stored on the lustre
%file system in NRAO Socorro.  From there it is ingested directly into
%the NRAO archive. 

%Sources that did not image acceptably are added to the failTargets
%list.  This is referenced in the HTML Report.

\clearpage

\end{document}
