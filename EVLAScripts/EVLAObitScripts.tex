\documentclass[11pt]{article}

\begin{document}

{
\begin{center}
{\Large Obit Development Memo 29\\ }
{\Large EVLA Continuum Scripts: \\
Outline of Data Reduction and Heuristics} \\
~ \\
Bill Cotton \\
\today
\end{center}}

\section{Introduction}



\subsection{Scope}

The scope of the present version of the EVLA continuum scripts is to
perform standard calibration and editing of EVLA data and produce
continuum, possibly wideband, images of target sources.
Logs, reports and numerous diagnostic plots help evaluate the results
of the processing.
If default processing parameters are adequate, the scripts will start
from EVLA archive ASDM/BDF files and result in FITS images,
calibrated data, reports, plots etc.
The scripting is also capable of being highly tuned to a particular
project and can be rerun in whole or part with user specified
parameters.

\subsection{Software}

The EVLA Obit scripts are:

\begin{itemize}
\item Written in python, and
\item Use Obit and AIPS tasks to do the data processing, and
\item Use AIPS data structures for intermediate data, and
\item Write FITS images and (AIPS FITAB format) calibrated datasets.
\end{itemize}

AIPS (http://www.aips.nrao.edu/index.shtml) and\\
Obit (http://www.cv.nrao.edu/~bcotton/Obit.html)
are installed on all NRAO Linux computers and available for 
installation via download to non-NRAO computers.
The EVLA scripts are in the \$OBIT/python directory with the template
parameter script in \$OBIT/share/scripts.

\section{Execution}
Several steps are needed to execute the EVLA scripts.
\subsection{Generate parameter scripts}
The processing is guided by values in python parameter scripts.
These scripts can be created and initialized by information gleaned from
the EVLA archive ALMA Science Data Model (ASDM) files using routine
EVLACal.EVLAPrepare (see section \ref{Details}).
This will create one or more parameter files, each of which needs to
be processed separately.

Alternatively, the parameter file can be derived manually using the
template file \$OBIT/share/scripts/EVLAContTemplateParm.py and making
the substitutions described in the file.

\subsection{Modify parameter scripts}
If the default values in the automatically generated parameter
script(s) are not appropriate, they can be changed.
The details of each processing step and the parameters used are
described in Section \ref{Details}.
Default parameters and control switches can be overridden in the
parameter scripts.
Additional calibrator model information can be entered as described in
section \ref{calmodel}.
The end of the parameter script contains switches to turn on and off
various stages.

\subsection{AIPS and Obit setup scripts}
A script needs to be created giving the details of the AIPS and Obit
installations. 
This script is described in detail in Section \ref{Setup}.

\subsection{Execute scripts}
Each of the parameter scripts can be executed from the Unix shell by
\begin{verbatim}
 > ObitTalk EVLAContPipe.py AIPSSetup.py \
      EVLAContParm_myProject_Cfg0\Nch64.py
\end{verbatim}
where EVLAContParm\_myProject\_Cfg0\_Nch64.py is the name of your
parameter script.

This procedure should start from an archive data set and result in a
set of calibrated data, images, reports, logs and various diagnostic
plots, see Section \ref{products} for details.
\section{Calibrator models}\label{calmodel}
The standard EVLA flux density calibrators are all resolved in the
more extended configurations and an accurate, wideband source model at
an appropriate frequency is needed for good calibration.
Calibrator source models using the Obit CLEAN components with spectra
are distributed in \$OBIT/share/data and should be copied to the first
FITS directory.

The EVLA calibration scripts operate on arrays of calibrator dict
structures with the following entries:
\begin{itemize}
\item{\tt  Source:}       Source name as given in the SU table. 
\item{\tt  CalName:}      Calibrator model Cleaned AIPS  map name 
\item{\tt  CalClass:}     Calibrator model Cleaned AIPS  map class
\item{\tt  CalSeq:}       Calibrator model Cleaned AIPS  map seq
\item{\tt  CalDisk:}      Calibrator model Cleaned AIPS  map disk
\item{\tt  CalNfield:}    Calibrator model No. maps to use for model
\item{\tt  CalCCVer:}     Calibrator model CC file version
\item{\tt  CalBComp:}     Calibrator model First CLEAN comp to use, 1/field
\item{\tt  CalEComp:}     Calibrator model Last CLEAN comp to use, 0=>all
\item{\tt  CalCmethod:}   Calibrator model Modeling method, 'DFT','GRID','    '
\item{\tt  CalCmodel:}    Calibrator model Model type: 'COMP','IMAG'
\item{\tt  CalFlux:}      Calibrator model Lowest CC component used
\item{\tt  CalModelSI:}   Calibrator Spectral index
\item{\tt  CalModelFlux:} Parameterized model flux density (Jy)
\item{\tt  CalModelPos:}  Parameterized model Model position offset (asec)
\item{\tt  CalModelParm:} Parameterized model Model parameters (maj, min, pa, type)
\end{itemize}
These dicts are created in the parameter script by routine
EVLACal.EVLACalModel for the various types of calibrators.
EVLACal.EVLAStdModel is then used to fill in the details about
calibrators it knows about and can find in the first FITS directory.
Information not known to these scripts may be entered into the
calibrator dict structure in the parameter script.

\section{AIPS and Obit Setup}\label{Setup}
These scripts use data in AIPS format and some AIPS tasks; the
location of the AIPS data directories and other details as well as the
Obit initialization are given in the AIPSSetup.py file.
The items that need to be specified are:
\begin{itemize}
\item {\tt adirs} \\
A list of the AIPS data directories as a tuple,  the first element is
the URL of the ObitTalkServer or None for local disk.
The second element is the directory path.
\item {\tt fdirs} \\
A list of the FITS data directories as a tuple,  the first element is
the URL of the ObitTalkServer or None for local disk.
The second element is the directory path.
\item {\tt user} \\
The AIPS user number to be used.
\item {\tt AIPS\_ROOT} \\
The root of the AIPS system directories.
An environment variable of this name is set by the AIPS startup scripts.
Python None will default to your AIPS setup.
\item {\tt AIPS\_VERSION} \\
The  AIPS version.
An environment variable of this name is set by the AIPS startup scripts.
Python None will default to your AIPS setup.
\item {\tt DA00} \\
The  AIPS DA00 directory (TDD000004; file needed).
An environment variable of this name is set by the AIPS startup
scripts.
Python None will default to your AIPS setup.
\item {\tt OBIT\_EXEC} \\
The root directory of your Obit directories.
Python None will default to your system installation on NRAO Linux machines.
\item {\tt noScrat} \\
A list of AIPS disks to avoid for scratch files, max. 10.
\item {\tt nThreads} \\
The maximum number of threads allowed to be used.
This generally should not be more than the number of cores available.
\item {\tt disk} \\
The AIPS disk number to use for temporary storage of the data and images.
\end{itemize}

An example AIPSSetup.py file follows, items which may need to be
modified are marked by {\tt <====}.:
\begin{verbatim}
#   <====  Define AIPS  and FITS disks
adirs = [(None, "/export/data_1/GOLLUM_1"),
         (None, "/export/data_1/GOLLUM_2"),
         (None, "/export/data_1/GOLLUM_3"),
         (None, "/export/data_1/GOLLUM_4"),
         (None, "/export/data_2/GOLLUM_5"),
         (None, "/export/data_2/GOLLUM_6"),
         (None, "/export/data_2/GOLLUM_7"),
         (None, "/export/data_2/GOLLUM_8")]
fdirs = [(None, "/export/users/aips/FITS")]

############################# Initialize OBIT #####################
err     = OErr.OErr()
user    = 104                      # <==== set user number
ObitSys = OSystem.OSystem ("Script", 1, user, 0, [" "], \
                         0, [" "], True, False, err)
OErr.printErrMsg(err, "Error with Obit startup")
# Setup AIPS
AIPS.userno = user
AIPS_ROOT    = "/home/AIPS/"        # <==== set root of AIPS
AIPS_VERSION = "31DEC12/"           # <==== set AIPS version
DA00         = "/home/AIPS/DA00/"   # <==== set AIPS DA00 directory
#  <====  Define OBIT_EXEC for access to Obit Software 
OBIT_EXEC    = "/export/data_1/obit/ObitInstall/ObitSystem/Obit/"
# setup environment
ObitTalkUtil.SetEnviron(AIPS_ROOT=AIPS_ROOT, AIPS_VERSION=AIPS_VERSION, \
                        OBIT_EXEC=OBIT_EXEC, DA00=DA00, ARCH="LINUX", \
                        aipsdirs=adirs, fitsdirs=fdirs)
# List directories
ObitTalkUtil.ListAIPSDirs()
ObitTalkUtil.ListFITSDirs()
noScrat     = []                   # <==== AIPS disks to avoid 
nThreads    = 6                    # <====  Number of threads allowed
disk        = 1                    # <====  AIPS disk number
\end{verbatim}

\section{The Process Overview}

The scripted processing uses the following processes.
Many of the default processing parameters are frequency dependent and
may be overridden and the various steps may be turned on or off.

The general approach to calibration and editing is to first apply
editing steps which can be applied to uncalibrated data to remove the
most serious RFI.
Then an initial pass at calibration is done and a pass at the editing
needing calibrated data.
Calibration aids in the editing as calibrator data with no detections
are effectively removed and calibration with deviant amplitude
solutions are also removed.
Once the first pass at editing and calibration is completed, the
initial calibration tables are deleted and the calibration repeated.
This procedure removes the bulk of the RFI infected and other bad
data.

Diagnostic plots at various stages of the processing are generated.
These  include plots of calibration results as well as sample spectra.

The calibrated data are then imaged, using wideband imaging if
appropriate. 
Images, calibrated data and calibration tables are saved to FITS files
and a number of source dependent diagnostic plots are generated.
Finally, an HTML report is generated allowing easy examination and
access to the various products.
A processing log is kept containing most details of the processing.

Following is a summary of the processing.
Details and parameters which may be modified are described in a
section \ref{Details}.

\begin{enumerate}
\item Generation of parameter scripts from ASDM
\item Data converted to AIPS format
\item Hanning if necessary
\item Clear previous calibration
\item Copy initial FG table 
\item Flag end channels 
\item Apply Special Editing
\item Quack
\item Shadow Flagging 
\item Initial Time domain flagging
\item Initial Frequency domain flagging
\item Initial RMS flagging of calibrators
\item Parallactic angle correction
\item Find reference antenna 
\item Plot raw sample spectra 
\item Delay calibration
\item Bandpass calibration
\item Amp \& phase Calibration 
\item Flagging of calibrated data
\item Recalibration 
\begin{enumerate}
\item Parallactic angle correction
\item Delay calibration
\item Bandpass calibration
\item Amp \& phase Calibration 
\item Flagging of calibrated data
\end{enumerate}
\item Calibrate and average data
\item Cross Pol clipping
\item R-L  delay calibration
\item Instrumental polarization calibration
\item R-L phase calibration
\item VPol clipping
\item Plot final calibrated spectra 
\item Image targets
\item Generate source report
\item Save images, calibrated data
\item Contour plots of images
\item source UV diagnostic plots
\item Generate HTML summary
\item Cleanup AIPS directories
\end{enumerate}

\section{Script Stage Details}\label{Details}
Details of the various processing steps and the parameters are
described in the following.
Processing parameters are stored in a python dict named parms and may
be specified in the parameter script as
\begin{verbatim}
parms["parameter"] = value
\end{verbatim}
Tables in the following give the parameter name, default value and a
description.
%Note: Not all stages are performed by default.

\begin{enumerate}
\item Generation of parameter scripts from ASDM\\
This step is performed from ObitTalk to generate the parameter
script(s).
\begin{verbatim}
>>> import EVLACal
>>> ASDMRoot = "/export/myData/12A-999.sb9332941.eb9360588.../"
>>> EVLACal.EVLAPrepare(ASDMRoot, err, project="myData")
\end{verbatim}
This will parse the ASDM indicated and generate a parameter script for
each configuration/number of channel combinations needed to process
all data.
Note: this will also include configurations used for calibration
purposes only, such as offset pointing, so use Obit task ASDMList to see
which configurations have useful data.
For each configuration/number of channel, a parameter script with name
of the form EVLAContParm\_$<$project$>$\_Cfg$<$config$>$\_Nch$<$no channels$>$.py
will be generated.

\begin{center}
\begin{tabular}{|l|c|l|}
\hline
ASDMRoot & & Root directory of ASDM/BDF data\\
project  & & Project name, 12 or fewer characters, \\
 & & used as AIPS file name \\
template & EVLAContTemplateParm.py & name of the parameter template
file\\
parmFile &  & Name of desired parameter file, \\
 & & generated if not given\\
\hline
\end{tabular}
\end{center}
%
\item General control parameters\\
These parameters control the naming of files, whether UV data is used
in compressed form and script debugging control.
\begin{center}
\begin{tabular}{|l|c|l|}
\hline
project  & ?? &  Project name, 12 or fewer characters,  used as AIPS file name\\
 session & ?? &  session code, generated from configuration and no. channels\\
 band  & ?? &  Observing band code, derived from ASDM frequencies\\
 Compress & False &  Use compressed UV data?\\
 check & False &  Only check script, don't execute tasks\\
 debug & False &  run tasks debug\\
  &  &  \\
\hline
\end{tabular}
\end{center}
%
\item Data converted to AIPS format\\
The bulk of the processing uses AIPS format UV data and images.
The ASDM/BDF data is first converted to an AIPS data file using Obit/BDFIn.
The details of the AIPS configuration are given in the AIPSSetup.py
file provided to the processing script.
\begin{center}
\begin{tabular}{|l|c|l|}
\hline
doLoadArchive  & True & Load AIPS data from archive? \\
archRoot  & ?? &  User specified to create parameter script\\
selConfig  & ?? &  Frequency configuration, generated from ASDM\\
seq     & 1 &  AIPS sequence number to use \\
selBand & ??  &  Data band-code, derived from ASDM\\
selChan & ??  &  Number of spectral channels, derived from ASDM\\
selNIF  & ??  &  Number of spectral windows (IFs), derived from ASDM\\
calInt  & ??  &  Calibration table interval (min), EVLA configuration
dependent\\
  &  & A: 0.3, B:0.45, C:1.0, D:2.0\\
doSwPwr & False  &  Apply gain corrections from Switched Power table\\
\hline
\end{tabular}
\end{center}
%
\item Hanning \\
At lower frequencies and compact configurations, RFI signals are
frequently sufficiently strong and narrow to cause serious ``Gibbs''
ringing due to the truncation of the lag spectra.
Hanning smoothing can be used to suppress this effect.
\begin{center}
\begin{tabular}{|l|c|l|}
\hline
doHann  &  &  Apply Hanning smoothing? EVLA configuration
 and frequency dependent\\
  &  & A:False, B:$\nu<$ 8 GHz, C:$\nu<$ 8 GHz, D:$\nu<$ 8 GHz\\
doDescm  & True  & If True, drop every other channel after smoothing  \\
\hline
\end{tabular}
\end{center}
%
\item Clear previous calibration\\
If the script is restarted it is frequently desirable that previous
attempts at calibration and editing be removed.
\begin{center}
\begin{tabular}{|l|c|l|}
\hline
doClearTab   & True & Clear cal/edit tables? \\
doClearGain  & True & Clear SN and CL tables $>$ 1? \\
doClearFlag  & True & Clear FG tables $>$ 1? \\
doClearBP    & True & Clear BP tables? \\
\hline
\end{tabular}
\end{center}
%
\item Copy initial FG table \\
To allow restarting of the flagging, the on-line flags which are in FG
table 1 are copied to table 2 and new flags added there.
This should be turned off if the script is restarted except at the beginning.
\begin{center}
\begin{tabular}{|l|c|l|}
\hline
doCopyFG  & True & Copy FG 1 to FG 2? \\
\hline
\end{tabular}
\end{center}
%
\item Flag end channels\\
The first and last few channels in each IF are flagged if FG table 2.
This can be turned off by setting BChDrop and EChDrop to 0.
\begin{center}
\begin{tabular}{|l|c|l|}
\hline
BChDrop  & ?? & Number of channels to flag at the beginning \\
 & & If $\nu<$ 8 GHz, max(2, 6.*(nchan/64)\\
 & & If $\nu>$ 8 GHz, 3\\
EChDrop  & ?? & Number of channels to flag at the end \\
 & & If $\nu<$ 8 GHz, max(2, 4.*(nchan/64)\\
 & & If $\nu>$ 8 GHz, 2\\
\hline
\end{tabular}
\end{center}
%
%
\item Apply Special Editing\\
If some data are known to be bad, e.g. no receiver, then this
information can be passed to the script.
if doEditList is True, each entry is a python dict with the following:
\begin{itemize}
\item{\tt timer:} The affected time range as a pair of strings of the
form day/hour:min:sec.
\item{\tt Ant:}A baseline specification as a pair of antenna numbers,
if the second is zero, then all baselines to the first antenna number
is flagged.  
If the first is zero, then all antennas are flagged
\item{\tt IFs:} Range (1-rel)  of IFs (spectral windows) to flag.
If the second is zero then all IFs higher than the first are flagged.
\item{\tt Chans:} Range (1-rel) of channels to flag.
If the second is zero then all channels higher than the first are flagged.
\item{\tt Stokes:} Array of flags, 1$=>$flag, $0=>$ not flag; in order
  RR, LL, RL, LR.
\item{\tt "Reason:} Up to 24 characters giving reason.
\end{itemize}
an example:
\begin{verbatim}
parms["doEditList"]  = True        # Edit using editList?
parms["editList"] = [
    {"timer":("0/00:00:0.0","5/00:00:0.0"),"Ant":[1,0],
     "IFs":[1,0],"Chans":[1,0],  "Stokes":'1111',"Reason":"No Rcvr"}
]
\end{verbatim}

\begin{center}
\begin{tabular}{|l|c|l|}
\hline
doEditList  & False & Edit using editList? \\
editFG      & 2     & Table to apply edit list to\\
 editList   & [ ]   & List of data to flag \\
  &  &  \\
\hline
\end{tabular}
\end{center}
%
\item Quack\\
Data at the beginning and end of each scan can be flagged using
Obit/Quack. 
\begin{center}
\begin{tabular}{|l|c|l|}
\hline
doQuack       & True    & Quack data? \\
quackBegDrop  & 0.1     & Time to drop from start of each scan (min) \\
quackEndDrop  & 0.0     & Time to drop from end of each scan (min) \\
quackReason   & "Quack" & Reason string \\
\hline
\end{tabular}
\end{center}
%
\item Shadow Flagging \\
In the more compact EVLA configurations, some antennas may shadow
others at times.
The affected data may be flagged using AIPS/UVFLG.
\begin{center}
\begin{tabular}{|l|c|l|}
\hline
doShad  &      & Do shadow flagging? Configuration Dependent \\
        &      & A:False, B:False, C:True, D:True\\
shadBl  & 25.0 &  Minimum shadowing baseline (m)\\
\hline
\end{tabular}
\end{center}
%
\item Initial Time domain flagging\\
Obit task MednFlag can be used to flag data by amplitudes deviant from
a running median by more than a specified amount.
This is performed independently on each data stream (baseline,
channel, IF, poln).
At this point that data are uncalibrated.
\begin{center}
\begin{tabular}{|l|c|l|}
\hline
doMedn       & True & Median editing? \\
mednSigma    & 10.0 &  Sigma clipping level\\
mednTimeWind & 1.0  &  Window width (min) for median flagging\\
mednAvgTime  & 10.0/60. & Averaging time (min) \\
mednAvgFreq  & 0    &  1$=>$avg mednChAvg chans, 2$=>$avg all chan,\\
 & & 3$=>$avg chan and IFs\\
mednChAvg    & 1    &  Number of channels to average\\
\hline
\end{tabular}
\end{center}
%
\item Initial Frequency domain flagging\\
The uncalibrated data can be examined for impulsive signals in
frequency by comparison with a running median in each spectrum and
deviant data are flagged using Obit task AutoFlag.
Since bandpass corrections have not been determined and applied at
this stage, structure in the instrumental bandpass will increase the
apparent RMS in the baseline reducing the sensitivity of this test.
\begin{center}
\begin{tabular}{|l|c|l|}
\hline
doFD1       & True &  Do initial frequency domain flagging?\\
FD1widMW    & 31   &  Width of the initial FD median window\\
FD1maxRes   & 5.0  &  Clipping level in sigma \\
FD1TimeAvg  & 1.0  &  time averaging (min). for initial FD flagging\\
\hline
\end{tabular}
\end{center}
%
\item Initial RMS flagging of calibrators \\
Calibrators are expected to be simple and have  significant SNR so can
be edited by having an RMS/average amplitude of less than some
amount.
Discrepant calibrator data can be flagged in this step using Obit task
AutoFlag.
\begin{center}
\begin{tabular}{|l|c|l|}
\hline
doRMSAvg   & True &  Edit calibrators by RMS/Avg?\\
RMSAvg     & 3.0  &  Max RMS/Avg for time domain RMS filtering\\
RMSTimeAvg & 1.0  &  Time averaging (min)\\
\hline
\end{tabular}
\end{center}
%
\item Parallactic angle correction\\
Phases are corrected for the effects of the parallactic angle using
Obit task CLCor.
The initial CL table is copied to CL 2 and modified.
\begin{center}
\begin{tabular}{|l|c|l|}
\hline
 doPACor & True &  Make parallactic angle correction?\\
\hline
\end{tabular}
\end{center}
%
%
\item Reference antenna\\
The choice of reference antenna is of some importance but nothing in
the ASDM helps in this choice.
In addition, at least early EVLA data may have data for antennas with
no receiver and such antennas are unsuitable for use as reference
antenna.
If the reference antenna is unspecified (0),
this step runs Obit task Calib on the bandpass calibrator(s) (assumed to
give good fringes) using the middle half of each spectrum.
The resultant SN table is then examined for the antenna with the
maximal amount of valid solutions and with the highest average SNR;
this antenna is used as the reference antenna.
\begin{center}
\begin{tabular}{|l|c|l|}
\hline
refAnt  &  & Reference antenna, if $<=$ 0 then determine \\
 BPCals &  & Determined from the ASDM \\
 bpsolint1 &  &  Bandpass first solution interval, \\
 & & configuration and  frequency dependent\\
  &  &  \\
\hline
\end{tabular}
\end{center}
%
\item Plot Raw spectra\\
At this point plots of sample spectra can be made to display residual
RFI and other problematic data,
\begin{center}
\begin{tabular}{|l|c|l|}
\hline
doRawSpecPlot  & True &  Plot diagnostic raw spectra?\\
plotSource     & & Default is first bandpass calibrator\\
plotTime       & & List of start and end time in days.\\
  &  & Default is first bandpass calibrator scan\\
refAnt  &  & Reference ant., baselines to refAnt are plotted \\
\hline
\end{tabular}
\end{center}
\newpage
%
\item Delay calibration\\
Parallel hand group delays are solved for using the list of
calibrator models in DCals.
Obit task Calib solves for the delays which are then smoothed and
applied to all sources in a new CL table using Obit task CLCal.
Solutions can be plotted as well as sample spectra applying the delay
calibration.
\begin{center}
\begin{tabular}{|l|c|l|}
\hline
doDelayCal  & True &  Determine/apply delays? \\
DCals  &  & The list of delay calibrators are determined from the ASDM, \\
  &  & all amplitude, phase and bandpass calibrator. \\
  &  & The list of models is determined from the parameter script \\
  & & using standard calibrator models. \\
delayBChan  &  &  first channel to use in delay solutions \\
  &  &  max(2, 0.05*nchan)\\
delayEChan  &  &  highest channel to use in delay solutions\\
  &  &  min(nchan-2, nchan-0.05*nchan\\
solInt  &  &  Solution interval (min), config. dependent\\
  &  &  A:2 sec, B: 5 sec, C:10 sec, D:15 sec.\\
refAnts  & [refAnt] & Delay reference ant., baselines to refAnt are plotted \\
doTwo  & True &  Use two baseline combinations in delay cal\\
delayZeroPhs &  True &  Zero phase in Delay solutions?\\
doSNPlot       & True &  Plot calibration solutions?\\
doSpecPlot     & True &  Plot diagnostic calibrated spectra?\\
plotSource     & & Default is first bandpass calibrator\\
plotTime       & & List of start and end time in days.\\
\hline
\end{tabular}
\end{center}
\newpage
%
\item Bandpass calibration\\
Bandpass calibration uses Obit task BPass and calibrator model list BPCals.
BPass does a two pass calibration, the first doing a phase only
calibration to straighten out the phases followed by a longer amplitude
and phase calibration using blocks of channels.
The resultant solutions are then combined into a BP table
\begin{center}
\begin{tabular}{|l|c|l|}
\hline
doBPCal  & True &  Determine/apply bandpass calibration? \\
BPCals  &  & The list of bandpass calibrators is determined from the ASDM, \\
  &  & The list of models is determined from the parameter script \\
  & & using standard calibrator models. \\
bpsolint1  &  &  BPass phase correction solution in min, frequency
dependent: \\
 & &  $\nu<1$ GHz: 10sec, L band:15 sec, S, C, Ku, K, Ka  band 10 sec,
\\
 & &  Q band 5 sec. \\
bpsolint2  & 10.0 & BPass bandpass solution interval (min) \\
bpsolMode  & 'A\&P'  &  Bandpass type 'A\&P', 'P', 'P!A'\\
bpBChan1  & 1 &  Low freq. channel,  initial cal\\
bpEChan1  & 0 &  Highest freq channel, initial cal, 0$=>$all\\
bpBChan2  & 1 &  Low freq. channel for BP cal\\
bpEChan2  & 0 &  Highest freq channel for BP cal,  0$=>$all \\
bpChWid2  & 1 &  Number of channels in running mean BP soln\\
bpDoCenter1  & None &  Fraction of  channels in 1st, overrides
bpBChan1, bpEChan1\\
bpUVRange  & [0.0,0.0] &  UV range for bandpass cal zeroes$=>$ all\\
refAnt  &  & BP reference ant., baselines to refAnt are plotted \\
doSpecPlot     & True &  Plot diagnostic calibrated spectra?\\
plotSource     & & Default is first bandpass calibrator\\
plotTime       & & List of start and end time in days.\\
\hline
\end{tabular}
\end{center}
\newpage
%
\item Amp \& phase Calibration \\
Standard flux density calibrators have their flux densities entered
into the SU table using Obit task SetJy, other calibrators have their
flux density entries set to 1.0.
All the amplitude and phase calibrators have Obit/Calib run using
their models and doing amplitude and phase solutions.
Solutions are then median window smoothed using Obit/SNSmo to time
solSmo clipping really wild points.
Obit task GetJy then solves for the flux densities for non flux
density calibrators and adjusts the SU and SN tables. If doAmpEdit is
True, solutions in each IF (spectral window) more than ampSigma from
the mean are flagged both in the SN table and in FG table ampEditFG.
Finally solutions are applied to the previous CL table to create a new
CL table.
Solution plots are written into file\\
parms["project"]+"\_"+parms["session"]+"\_"+parms["band"]+"APCal.ps".
\begin{center}
\begin{tabular}{|l|c|l|}
\hline
doAmpPhaseCal  & True & Do amplitude and phase calibration? \\
ACals  &  & The list of amplitude calibrators are determined from the ASDM, \\
  &  & The list of models is determined from the parameter script \\
  & & using standard calibrator models. \\
PCals  &  &  The list of phase calibrators are determined from the ASDM\\
refAnt  &  & Reference antenna \\
solInt  &  &  Solution interval (min), config. dependent:\\
  &  & A: 2 sec, B: 5 sec, , C:  10 sec, D: 15 sec.\\
ampBChan  &  & first channel to use in A\&P solutions \\
  &  &   max(2, 0.05*nchan)\\
ampEChan  &  &  highest channel to use in A\&P solutions\\
  &  &  min(nchan-2, nchan-0.05*nchan)\\
solSmo  & 0.0 &  Smoothing interval for Amps (min)\\
ampScalar  & False &  Ampscalar solutions?\\
doAmpEdit  & True  & Edit/flag on the basis of amplitude solutions \\
ampSigma  & 20.0 &  Multiple of median RMS about median gain to clip/flag\\
ampEditFG  & 2 &  FG table for editing \\
doSNPlot       & True &  Plot calibration solutions?\\
\hline
\end{tabular}
\end{center}
\newpage
%
\item Flagging of calibrated data\\
Calibrated data are then edited using Obit/AutoFlag. 
If recalibration is to be done, then this is only on the calibrators,
else all sources.
Data with amplitudes outside of a given range are flagged and data
overly discrepant from a running median in frequency is flagged.
\begin{center}
\begin{tabular}{|l|c|l|}
\hline
doAutoFlag  & True &  Autoflag editing after first pass calibration?\\
IClip  & ?? &  AutoFlag Stokes I clipping\\
 & & 200 Jy for $\nu> 1$ GHz, else 20000. \\
minAmp  & 1.0e-5 & Minimum allowable amplitude \\
timeAvg  & 0.33 &  AutoFlag time averaging in min.\\
doAFFD  & True & do AutoFlag frequency domain flag \\
FDmaxAmp  & IClip[0] &  Maximum average amplitude (Jy)\\
FDmaxV  & VClip[0] &  Maximum average VPol amp (Jy)\\
FDwidMW  & 31 & Width of the median window \\
FDmaxRMS  & [5.0,.1] &  FDmaxRMS\\
FDmaxRes  & 6.0  &  Max. residual flux in sigma\\
FDmaxResBL  & 6.0 &  Max. baseline residual\\
FDbaseSel  & [0,0,0,0] & Channels for baseline fit \\
\hline
\end{tabular}
\end{center}
%
\item Recalibration \\
If doRecal is true then the previous calibration tables are deleted
and the calibration redone using the flag table from the first pass.
\begin{enumerate}
\item Parallactic angle correction\\
As before if doDelayCal2 = True.
\item Delay calibration\\
As before if doDelayCal2 = True.
\item Bandpass calibration\\
As before if doBPCal2 = True.
\item Amp \& phase Calibration \\
As before if doAmpPhaseCal2 = True.
\item Flagging of calibrated data\\
As before if doAutoFlag2 = True.
\end{enumerate}
\item Calibrate and average data\\
The calibration and editing files are then applied with possible
averaging in time and/or frequency.
This uses Obit/Splat which writes a multi-source file.
\begin{center}
\begin{tabular}{|l|c|l|}
\hline
doCalAvg  & True &  Calibrate and average?\\
avgClass  & "UVAvg" & AIPS class of calibrated/averaged UV data \\
seq  & 1 &  AIPS sequence \\
CalAvgTime  &  &  Time for averaging calibrated UV data (min), config
dependent: \\
 & & A: 1 sec, B:3 sec, C: 10 sec, D: 20 sec. \\
avgFreq  & 0 &  $0=>$ no averaging, $1=>$avg chAvg chans, $2=>$avg all
chan, \\
 & &$3=>$avg chan and IFs\\
chAvg  & 1 & Number of channels to average \\
CABChan& 1 &  First channel to copy \\
CAEChan& 0 &  Highest channel to copy, $0=>$ all higher than CABChan\\
CABIF  & 1 &  First IF to copy \\
CAEIF  & 0 &  Highest IF to copy, $0=>$ all higher than CABIF\\
Compress & False &  Write compressed UV data?\\
\hline
\end{tabular}
\end{center}
%
\item Cross Pol clipping\\
if XClip[0] is not None,  cross polarized data with amplitudes $>$ XClip[0] are flagged.
\begin{center}
\begin{tabular}{|l|c|l|}
\hline
XClip  & [5.0,0.05] & AutoFlag cross-pol clipping, None$=>$ no flagging \\
\hline
\end{tabular}
\end{center}
%
\item R-L  delay calibration\\
Determine right--left delays from a list of calibrators with R--L
phase and rotation measure.
The function EVLACal.EVLAPrepare will search for sources in the Field
table whose positions are those of calibrators with known position
angle (and presumably reasonable strength polarized components);
3C286 is an example.
These calibrators are added to the R-L calibrator list (parms["RLDCal"]
in the parameter script).
If the first source in this list is not None, and the data contains
the RL and LR correlations, the R-L delay calibration is performed. \\
%This is not yet supported in the automated scripting.
\begin{center}
\begin{tabular}{|l|c|l|}
\hline
doRLDelay   & ?? &  Do R-L delay calibration?\\
RLDCal      & ?? &  Array of triplets of (name, R-L phase (deg at 1 GHz), \\
            &    &  RM (rad/m**2)) for polarization angle calibrators\\
rlBChan     & 1            & First (1-rel) channel number\\
rlEChan     & 0            & Highest channel number. $0=>$ high in data. \\
rlUVRange   &  [0.0,0.0]   & Range of baseline used in kilo wavelengths, zeros=all\\
rlCalCode   & '  '         & Calibrator code\\
rlDoCal     & 2            & Apply calibration table? positive$=>$calibrate\\
rlgainUse   & 0            & CL/SN table to apply, $0=>$highest\\
rltimerange & [0.0,1000.0] & Time range of data (days)\\
%rlDoBand    & 1            & If $> 0$ apply bandpass calibration \\
%rlBPVer     & 0            & BP table to apply, $0=>$highest\\
rlflagVer   & 2            & FG table version to apply \\
rlrefAnt    & 0            & Reference antenna, defaults to refAnt\\
\hline
\end{tabular}
\end{center}
%
\item Instrumental polarization calibration\\
Determine instrumental polarization from a list of calibrators.
The function EVLACal.EVLAPrepare sets this list to those in the Delay
calibrator list.
Parameter parms["doPolCal"] is set True if this list is not empty and
the calibration performed if the data contains the RL and LR
correlations. 
Calibration uses Obit task PCal which determines antenna and source
polarization parameters on blocks of channels in a running window.
The antenna parameters are the ellipticity and orientation of the
feed; see Obit Development Memo 30 for details.\\
%This is not yet supported in the automated scripting and AIPS/PCAL is
%likely to fail.
%The old, single solution per IF is currently (May 2012) implemented
%as the most likely to work.
\begin{center}
\begin{tabular}{|l|c|l|}
\hline
doPolCal  & ??     &  Determine instrumental polarization? \\
PCInsCals & ??     &  Instrumental poln cals, name or list of names\\
PCFixPoln & False  &  If True, don't solve for source polarization\\
PCpmodel  & [0.,0.,0.,0.,0.,0.,0.] &  Instrumental poln cal source poln model.\\
PCAvgIF   & False  &  If True, average IFs in ins. cal.\\
PCSolInt  & 2.0    &  Instrumental solution interval (min), \\
          &        & $0=>$ scan average(?) \\
PCRefAnt  & 0      &  Reference antenna, defaults to refAnt\\
PCSolType & "    " &  Solution type, "LM  " (better), "    " (faster)\\
PCChInc   & 5      &  Channel step in spectrum \\
PCChWid   & 5      &  Number of channels to average \\
doPol     & False  &  Apply polarization cal in subsequent calibration?\\
PDVer     & 1      &  Apply PD table in subsequent polarization cal?\\
\hline
\end{tabular}
\end{center}
%
\newpage
\item R-L phase calibration\\
Determine R-L phase bandpass for each channel using Obit/RLPass
and calibrators in RLDCal.
The function EVLACal.EVLAPrepare will search for sources in the Field
table whose positions are those of calibrators with known position
angle (and presumably reasonable strength polarized components);
3C286 is an example.
These calibrators are added to the R-L calibrator list (parms["RLDCal"]
in the parameter script).
If the first source in this list is not None, and, the data contains
the RL and LR correlations, the R-L phase calibration is performed. 

RLPass uses a two pass calibration, the first using Stokes I and a
phase only self calibration to remove phase fluctuations; the second
pass fits the R-L phase in each running block of channels.
%Corrections are applied to the old BP table and a new BP table is created.
%Then IFs are adjusted using calibrator RLPCal, by imaging in each IF
%and forcing the summed Q and U components to the appropriate
%polarization angle. 
%Corrections are made  in the CL table and SU tables by AIPS/CLCOR.
A spectral plot of the  first RLDCal calibrators RL and LR data are made in file\\
parms["project"]+"\_"+parms["session"]+"\_"+parms["band"]+"RLSpec2.ps"\\
%This is not yet supported in the automated scripting.
\begin{center}
\begin{tabular}{|l|c|l|}
\hline
doRLCal  & ??  & Determine R-L phases? \\
RLDCal   & ??  &  Array of triplets of (name, R-L phase (deg at 1 GHz), \\
 & &  RM (rad/m**2)) for polarization BP calibrators\\
 & &  If None then no R-L delay recalibration \\
rlBChan     & 1            & First (1-rel) channel number\\
rlEChan     & 0            & Highest channel number. $0=>$ high in data. \\
rlUVRange   &  [0.0,0.0]   & Range of baseline used in kilo wavelengths, zeros=all\\
rlCalCode   & '  '         & Calibrator code\\
rlDoCal     & 2            & Apply calibration table? positive$=>$calibrate\\
rlgainUse   & 0            & CL/SN table to apply, $0=>$highest\\
rltimerange & [0.0,1000.0] & Time range of data (days)\\
rlDoBand    & 1            & If $> 0$ apply bandpass calibration \\
rlBPVer     & 0            & BP table to apply, $0=>$highest\\
rlflagVer   & 2            & FG table version to apply \\
rlrefAnt    & 0            & Reference antenna, defaults to refAnt\\
%RLPCal      & None         & Polarization angle calibrator, None$=>$ skip. \\
%RLPhase     & 0.0          & R-L phase of RLPCal (deg) at 1 GHz\\
%RLRM        & 0.0          & R-L calibrator RM (NYI)\\
rlChWid     & 3            & Number of channels in running mean RL BP soln\\
rlsolint1   & 10./60       & First solution interval (min), $0=>$ scan average\\
rlsolint2   & 10.0         & Second solution interval (min)\\
%rlCleanRad  & None         & CLEAN radius about center or None=autoWin\\
%rlFOV"      & 0.05         & Field of view radius (deg) needed to image RLPCal\\
doPol       & False        & Apply polarization cal?\\
PDVer       & -1           & Apply PD table?\\
doSpecPlot  & True         & If True make spectral plot.\\
\hline
\end{tabular}
\end{center}
%
\item VPol clipping\\
If VClip is not None, data with circularly polarized amplitudes $>$
VClip[0] are flagged. 
\begin{center}
\begin{tabular}{|l|c|l|}
\hline
VClip  & [2.0,0.05] & AutoFlag cross-pol clipping, None$=>$ no flagging  \\
  &  &  \\
\hline
\end{tabular}
\end{center}
%
\item Plot final calibrated spectra\\
At this point, plots of sample spectra can be made to display
calibrated data.
\begin{center}
\begin{tabular}{|l|c|l|}
\hline
doSpecPlot     & True &  Plot diagnostic spectra?\\
plotSource     &      & Default is first bandpass calibrator\\
plotTime       &      & List of start and end time in days.\\
refAnt         &      & Reference ant., baselines to refAnt are plotted \\
\hline
\end{tabular}
\end{center}
%
\item Image targets \\
All targets are imaged and deconvolved using Obit/Imager or
Obit/MFImage if wideband imaging needed (fractional spanned bandwidth
$\ge$ MBmaxFBW).
Phase only and amp and phase calibration may be applied if sources
exceed given thresholds.
If wideband imaging is used, then the resultant images are cubes
having planes:
\begin{enumerate}
\item Total intensity at reference frequency.
\item Spectral index at reference frequency
\item any higher order planes
\item One plane for each of the coarse frequency samples.
\end{enumerate}
\begin{center}
\begin{tabular}{|l|c|l|}
\hline
doImage     & True     & Image targets? \\
targets     & [?]       & Target list set from ASDM, empty$=>$all\\
seq         & 1        & AIPS sequence for images \\
doPol       & True     & Apply polarization cal?\\
PDVer       &  1       & Apply PD table?\\
outIclass   & "IClean" & Image AIPS class\\
Stokes      & "I"      & Stokes to image \\
Robust      & 0.0      & Weighting robust parameter\\
FOV         &          & Field of view radius in deg, average $\nu$ dependent:\\
            &          & $\nu<1$ GHz: FWHM, L,S,C,X,Ku band: $0.5\times$FWHM\\
            &          & K,Ka,Q Band: $0.25\times$FWHM\\
Niter       & 500      & Max number of CLEAN iterations\\
minFlux     & 0.0      & Minimum CLEAN flux density (Jy) \\
minSNR      & 4.0      & Minimum Allowed SNR in self cal\\
maxPSCLoop  & 1        & Max. number of phase self cal loops\\
minFluxPSC  & 0.05     & Min flux density peak for phase self cal\\
solPInt     &          & Phase self cal solution interval (min), $\nu$ dependent \\
            &          &       $\nu<1$ GHz, L,C,X,Ku,K,Ka: 0.25, Q band:0.10\\
solPMode    & "DELA"   & Delay solution for phase self cal\\
solPType"   & "    "   & Solution type for phase self cal\\\
maxASCLoop  & 1        & Max. number of Amp+phase self cal loops\\
minFluxASC  & 0.5      & Min flux density peak for amp+phase self cal\\
solAInt     &          & amp+phase self cal solution interval (min),
$\nu$ dependent\\
            &          & $\nu<1$ GHz, L,C,X,Ku,K,Ka,Q: 3.0\\
solAMode    & "A\&P"   & Amp and phase self cal \\
solAType    & "    "   & Solution type for Amp and phase self cal\\
avgPol      & True     & Average poln in self cal?\\
avgIF       & False    & Average IF in self cal?\\
nTaper      & 0        & Number of additional imaging multi-resolution tapers\\
Tapers      & [20.0,0.0] &  List of tapers in pixels\\
do3D        & False    & Make ref. pixel tangent to celest. sphere for each facet\\
noNeg       & False    & Allow negative components in self cal model?\\
BLFact      & 1.01     & Baseline dependent time averaging for $>1.0$?\\
BLchAvg     & True     & Baseline dependent frequency averaging?\\
doMB        &  ??      & Set in parameter script depending on spanned bandwidth\\
MBnorder    & 1        & Order of wideband imaging \\
MBmaxFBW    & 0.05     & max. MB fractional bandwidth\\
CleanRad    & None     & CLEAN radius about center or None=autoWin\\
\hline
\end{tabular}
\end{center}
%
\newpage
\item Generate report\\
\begin{center}
\begin{tabular}{|l|c|l|}
\hline
doReport    & True     & Generate source report? \\
targets     & [?]      & Target list set from ASDM, empty$=>$all\\
seq         & 1        & AIPS sequence for images \\
outIclass   & "IClean" & Image AIPS class\\
Stokes      & "I"      & Stokes imaged \\
\hline
\end{tabular}
\end{center}
%
\item Save images, calibrated data\\
Images and calibrated/averaged data and calibration tables are written
to FITS files.
File names begin with \\
parms["project"]+parms["session"]+parms["band"]
followed by \\
$<$source\_name$>$+$<$Stokes$>$+"Clean.fits" for images and
"Cal.uvtab" for calibrated data and "CalTab.uvtab" for calibration
tables from the original data.
\begin{center}
\begin{tabular}{|l|c|l|}
\hline
doSaveImg & True &  Save target images to FITS?\\
targets   & [?]  & Target list set from ASDM, empty$=>$all\\
doSaveUV  & True & Save calibrated UV data for AIPS/FITAB format? \\
doSaveTab & True &  Save calibration tables for AIPS/FITAB format?\\
\hline
\end{tabular}
\end{center}
%
\item Contour plots of images\\
Contour plots are generated for target images.
Plot names are \\
parms["project"]+"\_"+parms["session"]+"\_"+parms["band"]
followed by the source name and ".cntr.ps" which are also converted
to jpeg with the suffix "jpg".
\begin{center}
\begin{tabular}{|l|c|l|}
\hline
doKntrPlots & True &  Generate contour plots?\\
targets   & [?]  & Target list set from ASDM, empty$=>$all\\
\hline
\end{tabular}
\end{center}
%
\item UV diagnostic plots\\
Plots of amplitude vs. baseline length, real vs. imaginary and UV
coverage are generated.
Plot names are \\
parms["project"]+"\_"+parms["session"]+"\_"+parms["band"]
followed by the source name and ".amp.ps",  ".ri.ps", or ".uv.ps"
which are also converted to jpeg with the suffix "jpg".
\begin{center}
\begin{tabular}{|l|c|l|}
\hline
doDiagPlots & True & Make UV diagnostic plots per source? \\
targets   & [?]   & Target list set from ASDM, empty$=>$all\\
\hline
\end{tabular}
\end{center}
%
\item Generate HTML Summary \\
Generate an HTML page with source statistics and links to the various
plots.
\begin{center}
\begin{tabular}{|l|c|l|}
\hline
doHTML  & True & Generate HTML reports? \\
\hline
\end{tabular}
\end{center}
%
\item Cleanup \\
AIPS data and image files are zapped.
\begin{center}
\begin{tabular}{|l|c|l|}
\hline
doCleanup  & True & Clean out AIPS directories? \\
\hline
\end{tabular}
\end{center}
%
\end{enumerate}

\section {The Products}\label{products}

\begin{itemize}
\item Calibrated (u,v) dataset with calibration and flagging tables in
AIPS FITAB format -- Tables from initial data and averaged
visibilities per input dataset. 
These files are\\
parms["project"]+parms["session"]+parms["band"]+"Cal.uvtab"
and parms["project"]+parms["session"]+parms["band"]+"CalTab.uvtab".
\item FITS Images -- for each target object in files\\
parms["project"]+"\_"+parms["session"]+"\_"+parms["band"]+\\
source\_name+".IClean.fits".\\
If wideband imaging is used, then the resultant images are cubes
having planes:
\begin{enumerate}
\item Total intensity at reference frequency.
\item Spectral index at reference frequency
\item any higher order planes
\item One plane for each of the coarse frequency samples.
\end{enumerate}
\item Diagnostic plots -- calibration and several per source.
The project plots have prefix
parms["project"]+"\_"+parms["session"]+"\_"+parms["band"] and are
\begin{itemize}
\item{\tt RawSpec.ps:} AIPS/POSSM plots of sample spectra with initial
editing but no calibration applied.
\item{\tt DelaySpec.ps:} AIPS/POSSM plots of sample spectra with initial
editing and delay calibration applied.
One set per pass through the calibration.
\item{\tt BPSpec.ps:} AIPS/POSSM plots of sample spectra with initial
editing and delay and bandpass calibration applied.
One set per pass through the calibration.
\item{\tt Spec.ps:} AIPS/POSSM plots of sample spectra with final
editing and calibration applied.
\item{\tt RLSpec2.ps:} AIPS/POSSM plots of sample RL and LR spectra
with final editing and calibration applied.
\item{\tt DelayCal.ps:} AIPS/SNPLT plots of delay calibration.
\item{\tt APCal.ps:} AIPS/SNPLT plots of amplitude and phase calibration.
\end{itemize}
The source plots have prefix
parms["project"]+"\_"+parms["session"]+"\_"+parms["band"] and are
\begin{itemize}
\item{\tt source\_name.cntr.jpg:} Source image contour plot as jpeg
\item{\tt source\_name.cntr.ps:} Source image contour plot as postscript
\item{\tt source\_name.amp.jpg:} Source amp. vs baseline plot as jpeg
\item{\tt source\_name.amp.pdf:} Source amp. vs baseline plot as pdf
\item{\tt source\_name.amp.ps:} Source amp. vs baseline plot as postscript
\item{\tt source\_name.ri.jpg:} Source real vs imaginary plot as jpeg
\item{\tt source\_name.ri.pdf:} Source real vs imaginary plot as pdf
\item{\tt source\_name.ri.ps:} Source real vs imaginary plot as postscript
\item{\tt source\_name.uv.jpg:} Source uv coverage plot as jpeg
\item{\tt source\_name.uv.pdf:} Source uv coverage plot as pdf
\item{\tt source\_name.uv.ps:} Source uv coverage plot as postscript
\end{itemize}
\item Reports and logs created during the process\\
The logfile is\\
parms["project"]+"\_"+parms["session"]+"\_"+parms["band"]+".log",
and the HTML report is\\
parms["project"]+"\_"+parms["session"]+"\_"+parms["band"]+".report.html".
%\item Meta-data for a VOTable to describe the products
\end{itemize}

The file set comprising all files and the meta-data are stored in a single
directory.  
%For approved pipeline use, this directory is stored on the lustre
%file system in NRAO Socorro.  From there it is ingested directly into
%the NRAO archive. 

%Sources that did not image acceptably are added to the failTargets
%list.  This is referenced in the HTML Report.

\clearpage

\end{document}
